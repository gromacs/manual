\documentclass[11pt,a4paper,twoside]{article}
\usepackage{here,picins,fancy,array,tabularx,multicol,dcolumn,makeidx,times,ifthen,enumitem,longtable,pdflscape,verbatim}

\begin{document}

\newcommand{\bfv}[1]{{\mbox{\boldmath{$#1$}}}}
% non-italicized boldface for math (e.g. matrices)                              
\newcommand{\bfm}[1]{{\bf #1}}
\newcommand{\dt}{\Delta t}
\newcommand{\rv}{\bfv{r}}
\newcommand{\vv}{\bfv{v}}
\newcommand{\F}{\bfv{F}}
\newcommand{\pb}{\bfv{p}}
\newcommand{\veps}{v_{\epsilon}}
\newcommand{\peps}{p_{\epsilon}}
\newcommand{\sinhx}[1]{\frac{\sinh{\left( #1\right)}}{#1}}
\newcommand{\fs}[1]{\begin{equation} \label{eqn:#1}}
\newcommand{\fe}{\end{equation}}
\newcommand{\p}{\partial}
\newcommand{\Bm}{\ve{B}}
\newcommand{\M}{\ve{M}}
\newcommand{\iM}{\M^{-1}}
\newcommand{\Tm}{\ve{T}}
\newcommand{\Sm}{\ve{S}}
\newcommand{\fo}{\ve{f}}
\newcommand{\con}{\ve{g}}
\newcommand{\lenc}{\ve{d}}

\setlength {\parindent}{0.0cm}
\setlength {\parskip}{1ex}
\newcommand{\ve}[1]{\mbox{\boldmath ${#1}$}} 
  % defines bold italic vectors. To be used in text or math mode.
  % Example: \ve{F}
\newcommand{\de}{\mbox{d}} 
  % defines a straight d for derivatives.
\newcommand{\intel}{Intel {\em i\/}860}
\newcommand{\gromacs}{GROMACS}
\newcommand{\gromosv}[1]{GROMOS-#1}
\newcommand{\gromos}{GROMOS}
\newcommand{\dline}{\hline\hline}
\newcommand{\etal}{{\em et al.}}
\newcommand{\beq}{\begin{equation}}
\newcommand{\eeq}{\end{equation}}
\newcommand{\bea}{\begin{eqnarray}}
\newcommand{\eea}{\end{eqnarray}}
\newcommand{\Dt}{{\Delta t}}
\newcommand{\half}{\frac{1}{2}}
\newcommand{\hDt}{\half \Dt}
\newcommand{\rvi}{\ve{r}_i}
\newcommand{\rvj}{\ve{r}_j}
\newcommand{\rvk}{\ve{r}_k}
\newcommand{\rij}{r_{ij}}
\newcommand{\rvij}{\ve{r}_{ij}}
\newcommand{\rnorm}{\frac{\rvij}{\rij}}
\newcommand{\Fvi}{\ve{F}_i}
\newcommand{\Fvj}{\ve{F}_j}
\newcommand{\Fvk}{\ve{F}_k}
\newcommand{\Fvij}{\ve{F}_{ij}}
\newcommand{\Fvji}{\ve{F}_{ji}}
\newcommand{\vvi}{\ve{v}_i}
\newcommand{\al}{\alpha}
\newcommand{\be}{\beta}
\newcommand{\ab}{\alpha\beta}
\newcommand{\rnij}{\ve{r}_{ij}^n}
\newcommand{\rni}{\ve{r}_i^n}
%\newcommand{\hdt}{\frac{\Delta t}{2}}
%\newcommand{\type}[1]{\\ {\tt \% #1}\\}
\newcommand{\normindex}[1]{#1{\index{#1}}}
\newcommand{\seeindexquiet}[2]{\index{#1|see{#2}}}
\newcommand{\seeindex}[2]{#1{\seeindexquiet{#1}{#2}}}
\newcommand{\swapindexquiet}[2]{{\index{#1 #2}}{\seeindexquiet{#2, #1}{#1 #2}}}
\newcommand{\swapindex}[2]{{\normindex{#1 #2}}{\seeindexquiet{#2, #1}{#1 #2}}}
\newcommand{\swapindexthreequiet}[3]{{\index{#1 #2 #3}}{\seeindexquiet{#2!#3!#1}{#1 #2 #3}}}
\newcommand{\swapindexthree}[3]{{\normindex{#1 #2 #3}}{\seeindexquiet{#2!#3!#1}{#1 #2 #3}}}
\newcommand{\pawsindexquiet}[2]{{\seeindexquiet{#1 #2}{#2, #1}}{\index{#2, #1}}}
\newcommand{\pawsindex}[2]{#1 #2{\pawsindexquiet{#1}{#2}}}
\newcommand{\boldindex}[1]{#1\index{#1@\textbf{#1}}}
\newcommand{\eg}{\em e.g.\@}
\newcommand{\ie}{\em i.e.\@}
\newcommand{\mc}[3]{\multicolumn{#1}{#2}{#3}}

% Commands for correct spacing in tables
%       TeX and TUG News, Vol.2, No.3, p10, 1993. 
%
\newcommand\T{\rule{0pt}{2.6ex}}         % Top strut
\newcommand\B{\rule[-1.2ex]{0pt}{0pt}}   % Bottom strut
\newcommand{\Ts}{\rule{0pt}{2.4ex}}      % Smaller top strut (To be used in 
                                         % math mode in \frac, together with 
                                         % \displaystyle

\newcommand{\captspace}{\vspace{2mm}}

% The total a4 paper width is 210 mm, and the margins are defined
% relative to the standard margin of 1 inch.
\setlength{\textwidth}{15cm}
\setlength{\oddsidemargin}{9.6mm}  %35mm-1inch
\setlength{\evensidemargin}{-0.4mm} %25mm-1inch
\setlength{\topmargin}{-4.4mm} % 21mm top margin
\setlength{\textheight}{22cm}

\renewcommand{\textfraction}{0.0}
\newcommand{\qtw}{3.75cm}  % 1/4 of the textwidth
\newcommand{\ttw}{4.33cm}  % 1/3 of the textwidth
\newcommand{\htw}{7.5cm}  % 1/2 of the textwidth
\newcommand{\ntw}{13cm} % 0.9 of the textwidth
\newcommand{\gmxmajor}{@MANUAL_MAJOR_VERSION@}
\newcommand{\gmxver}{@MANUAL_VERSION@}
\newcommand{\figref}[1]{Fig.~\ref{fig:#1}}
\newcommand{\figsref}[2]{Figs.~\ref{fig:#1} and ~\ref{fig:#2}}
\newcommand{\tabref}[1]{Table~\ref{tab:#1}}
\newcommand{\eqnref}[1]{eqn.~\ref{eqn:#1}}
\newcommand{\eqnsref}[2]{eqns.~\ref{eqn:#1} and \ref{eqn:#2}}
\newcommand{\secref}[1]{sec.~\ref{sec:#1}}
\newcommand{\tsecref}[1]{\ref{sec:#1}}
\newcommand{\ssecref}[1]{\ref{subsec:#1}}
\newcommand{\sssecref}[1]{\ref{subsubsec:#1}}
\newcommand{\chref}[1]{chapter~\ref{ch:#1}}
\newcommand{\appref}[1]{Appendix~\ref{app:#1}}
\newcommand{\wwwpage}{\href{http://www.gromacs.org}{www.gromacs.org}}
% NOTE: \wwwpage is explicitly included in the 'verbatim' citing instructions!
\newcommand{\email}{\href{mailto:gromacs@gromacs.org}{gromacs@gromacs.org}}
\newcommand{\mcc}[2]{\multicolumn{#1}{c|}{#2}}
\newcommand{\mcl}[2]{\multicolumn{#1}{l}{#2}}
\setlength{\headwidth}{\textwidth}
\setlength{\headheight}{1.0cm}

\section{Outline of Proposal for reorganizing the integrators}

\begin{itemize}
\item Organize integrator functionality with Monte Carlo moves as the base move
\item 
   \begin{itemize}
   \item MD is simply a molecular dynamics set with 100% acceptance
   \item More complicated MC moves can be added easily, including multistage
   \end{itemize}
\item MD integrators organized around Trotter factorization
  \begin{itemize}
  \item all MD moves implemented as Trotter factorizations
  \item keeping velocity and position incrementors separate will simplify code
  \item There is no ``leapfrog'' per se; the benefits of leapfrog are
    obtained by using average kinetic energies and concatenating
    constraint and velocity steps.
\end{itemize}
\item Two (and more) way splitting of the potential for long time scale integration
    \begin{itemize} 
      \item Allow more general ways to break apart long and short range interactions 
      \item Three part decomposition splitting for bonded interactions.
    \end{itemize}
\item Thermostats
  \begin{itemize} 
      \item Keep all thermostats, except for NH chains
      \item Replace NH chains with Langevin dynamics on the path
      \item Get Andersen working for velocity Verlet.
      \item Stochastic integrators replaced with OVRVO methods of
        Sivak et al. (http://arxiv.org/pdf/1301.3800v2.pdf)
      \item Both leapfrog and velocity Verlet versions implemented (when they are actually different)
  \end{itemize}
\item Barostats 
  \begin{itemize}
     \item Keep Berendsen, Parrinello-Rahman, MTTK 
     \item Figure out how to merge the PR and MTTK code, since they are very similar underlying algorithms.
     \item Ditch iteration for constraints for MTTK: It's just not
       worth it. Include approximations where necessary (as PR does
       now).
     \item Change MTTK be fully anisotropic (much easier w/o constraints).  
     \item Add a MC barostat
   \end{itemize}
\item Monte Carlo moves 
  \begin{itemize}
    \item We can randomly select between MC moves (obeys detailed balance)
      or select them in order (which obeys balance).
    \item Implement random particle translations as an example to others.
      There are other people who are willing to implement.
    \item Torsion moves may also be useful, but probably only viable in implicit solvent.
   \item Eventually include force routine that calculates energies only (not required at first)
   \end{itemize}
\item General Ensemble Moves
  \begin{itemize}
      \item These include expanded ensemble, replica exchange, simulated tempering
      \item Very little change required to support these
      \item CONSIDER dynamics in the general ensemble space.
   \end{itemize}
\end{itemize}
\section{First Steps}

\begin{enumerate}
\item The first step is to remove all iterative steps.  A draft has already
been completed in the master branch.
\item The next step is to make integrators explicitly Trotter
  factorization, making leapfrog code branch ``doubling'' of Trotter
  steps, starting with the temperature control, which is generally
  straightforward.  This process is underway.
\item The third step is to combine the pressure control integrators.
\item The next step is to Monte Carlo-ize the integrator routine. 
\end{enumerate}


\section{Basic organization}

The most general way to think of an integrator is as an instance of
Markov chain Monte Carlo sequence.
\begin{enumerate}
\item Generate a new configuration, given an old configuration, using some set of rules
\item Compare the new and old energies, and decide whether to change
  to a new configuration, or to discard it and remain with the old
  configuration.
\end{enumerate}

In standard Metropolis Monte Carlo, the decision is a based on the
energy difference between the two distributions:

\[P_{acc} = \min\{1,\exp(\beta U_{\mathrm{old}} - U_{\mathrm{new}}) \]

If the new energy is lower, we accept, if the new energy is higher, we
accept with some probability. We can potentially include weighting in
this acceptance ratio if we used a method to generate configurations
that is biased.

For molecular dynamics, then the generation of a new configuration is
one or more MD steps, and the acceptance rule is to accept with 100\%
probability.

One can perform hybrid MD by using a Metropolis Monte Carlo step after
dynamics.  If one is using good MD parameters, then the acceptance
rate will be nearly, but not exactly, 100\%; perfect NVE dynamics will
result in 100\% acceptance, because the energy difference will be
zero.  If one uses large time steps, then acceptance will be lower,
but a distribution with the correct distribution will result.  There
are a number of implementation details for hybrid MD, which we will be
outlined later.

In this way, only one high level integration scheme needs to be
implemented; all other methods can be slotted in trivially.

There are a number of algorithms that use properties of the underlying
shadow Hamiltonian, that can give better hybrid MC performance, but
that can be examined later.

\section{Trotter decomposition}

\subsection{Understanding reversible integrators: The Trotter decomposition}

The general formalism we can use to construct symplectic integrators
is the reversible Trotter decomposition formulation of dynamics.  This
can be used o understand the relationship between velocity Verlet and
leap-frog integration, as well as easily implement of thermostats and
barostats in {\gromacs}.

A system of coupled, first-order differential equations can be evolved
from time $t = 0$ to time $t$ by applying the evolution operator \bea
\Gamma(t) &=& \exp(iLt) \Gamma(0) \nonumber \\ iL &=&
\dot{\Gamma}\cdot \nabla_{\Gamma}, \eea where $L$ is the Liouville
operator, and $\Gamma$ is the multidimensional vector of independent
variables (positions and velocities).  

A short-time approximation to the true operator, accurate at time $\Dt
= t/P$, is applied $P$ times in succession to evolve the system as
\beq \Gamma(t) = \prod_{i=1}^P \exp(iL\Dt) \Gamma(0) \eeq 
For NVE dynamics, the Liouville operator is 
\bea iL = \sum_{i=1}^{N} \vv_i
\cdot \nabla_{\rv_i} + \sum_{i=1}^N \frac{1}{m_i}\F(r_i) \cdot
\nabla_{\vv_i}.  \eea 
This can be split into two additive operators
\bea iL_1 &=& \sum_{i=1}^N \frac{1}{m_i}\F(r_i) \cdot \nabla_{\vv_i}
\nonumber \\ iL_2 &=& \sum_{i=1}^{N} \vv_i \cdot \nabla_{\rv_i} \eea
Then a short-time, symmetric, 
and thus reversible approximation of the
true dynamics will be 
\bea \exp(iL\Dt) = \exp(iL_2\hDt) \exp(iL_1\Dt)
\exp(iL_2\hDt) + \mathcal{O}(\Dt^3).
\label{eq:NVE_Trotter}
\eea
This corresponds to velocity Verlet integration.  The first
exponential term over $\hDt$ corresponds to a velocity half-step, the
second exponential term over $\Dt$ corresponds to a full velocity
step, and the last exponential term over $\hDt$ is the final velocity
half step.  For future times $t = n\Dt$, this becomes
\bea
\exp(iLn\Dt) &\approx&  \left(\exp(iL_2\hDt) \exp(iL_1\Dt) \exp(iL_2\hDt)\right)^n \nonumber \\
             &\approx&  \exp(iL_2\hDt) \bigg(\exp(iL_1\Dt) \exp(iL_2\Dt)\bigg)^{n-1} \nonumber \\
             &       &  \;\;\;\; \exp(iL_1\Dt) \exp(iL_2\hDt) \nonumber \\
             &\approx&  \exp(iL_2\hDt) \bigg(\exp(iL_1\Dt) \exp(iL_2\Dt)\bigg)^{n} \exp(-iL_2\hDt)\nonumber 
\eea
This formalism allows us to easily see the difference between the
different flavors of Verlet integrators.  The leap-frog integrator can
be seen as starting with Eq.~\ref{eq:NVE_Trotter} with the
$\exp\left(iL_1 \dt\right)$ term, instead of the half-step velocity
term, yielding
\bea 
\exp(iLn\dt) &=& \exp\left(iL_1 \dt\right) \exp\left(iL_2 \Dt \right) + \mathcal{O}(\Dt^3).
\eea 
However, start at time $t-\hDt$, so the full step in velocity is between $t-\hDt$ and $t+\hDt$,
since it is a combination of the velocity half steps in velocity
Verlet. For future times $t = n\Dt$, this becomes
\bea 
\exp(iL\dt) &\approx& \bigg(\exp\left(iL_1 \dt\right) \exp\left(iL_2 \Dt \right)  \bigg)^{n}.
\eea 
Although at first this does not appear symmetric, as long as the full velocity
step is between $t-\hDt$ and $t+\hDt$, then this is simply a way of
starting velocity Verlet at a different place in the cycle.

Even though the trajectory and thus potential energies are identical
between leap-frog and velocity Verlet, the kinetic energy and
temperature will not necessarily be the same.  Standard velocity
Verlet uses the velocities at the $t$ to calculate the kinetic energy
and thus the temperature only at time $t$; the kinetic energy is then a sum over all particles
\bea
KE_{\mathrm{full}}(t) &=& \sum_i \left(\frac{1}{2m_i}\ve{v}_i(t)\right)^2 \nonumber\\ 
      &=& \sum_i \frac{1}{2m_i}\left(\frac{1}{2}\ve{v}_i(t-\hDt)+\frac{1}{2}\ve{v}_i(t+\hDt)\right)^2,
\eea
with the square on the {\em outside} of the average.  Standard
leap-frog calculates the kinetic energy at time $t$ based on the
average kinetic energies at the timesteps $t+\hDt$ and $t-\hDt$, or
the sum over all particles
\bea
KE_{\mathrm{average}}(t) &=& \sum_i \frac{1}{2m_i}\left(\frac{1}{2}\ve{v}_i(t-\hDt)^2+\frac{1}{2}\ve{v}_i(t+\hDt)^2\right),
\eea
where the square is {\em inside} the average.

We can see that the interpretation of kinetic energies as averages of
two half step velocities or averages of full step velocities is not
inherent to either leapfrog or velocity Verlet---instead, it is an
independent choice.

Currently (4.6), a non-standard variant of velocity Verlet which
averages the kinetic energies $KE(t+\hDt)$ and $KE(t-\hDt)$, exactly
like leap-frog, is implemented in {\gromacs} (as {\tt .mdp} file
option {\tt md-vv-avek} (though only working in single core mode).
Without temperature and pressure coupling, velocity Verlet with
half-step-averaged kinetic energies and leap-frog will be identical up
to numerical precision.  For temperature- and pressure-control
schemes, however, velocity Verlet with half-step-averaged kinetic
energies and leap-frog will be different, as will be discussed in the
section in thermostats and barostats.

The half-step-averaged kinetic energy and temperature are slightly
more accurate for a given step size. To be precise, the difference in
average kinetic energies using the half-step-averaged kinetic energies
({\em md} and {\em md-vv-avek}) will be closer to the kinetic energy
obtained in the limit of small step size than will the full-step
kinetic energy (using {\em md-vv}).  For NVE simulations, this
difference is usually not important, even though the energies will be
different.  Since the positions and velocities of the particles are
still identical; it makes a difference in the way the the temperature
of the simulations are {\em interpreted}, but {\em not} in the
trajectories that are produced.  

There is an interesting difference in the two kinetic energies;
although the kinetic energy is less biased with the half-step-averaged
method, meaning that it changes less as the timestep gets large, it is
also more noisy.  The RMS deviation of the total energy of the system
(sum of kinetic plus potential) in the half-step-averaged kinetic
energy case will be higher (about twice as high in most cases) than
the full-step kinetic energy.  The drift from the ``true'' trajectory
will still be the same, however, as again, the trajectories are
identical.

For NVT simulations, however, there {\em will} be a difference, since
the velocities of the particles are adjusted such that kinetic
energies of the simulations, which can be calculated either way, reach
the distribution corresponding to the set temperature.  In this case,
the three methods will not give identical results.

Because the velocity and position are both defined at the same time
$t$ the velocity Verlet integrator can be used for some methods,
especially rigorously correct pressure control methods, that are not
actually possible with leap-frog.  For pressure control simulations
where the fine details of the thermodynamics are important, only
velocity Verlet allows the true ensemble to be calculated.  In either
case, simulation with double precision may be required to get fine
details of thermodynamics correct.

We propose in 5.0 to explicitly separate the velocity half-step from
position step, even for leapfrog. This will cause a very slight
increase in computational cost, since the integration loop will have
to be traversed multiple times for each atom.  However, it will make
it much easier to mix and match integrators.  We can choose where to
actually sum the kinetic energy; it will only depend on where we want
to report energies, thus all-to-all communications will not be
affected.

\subsubsection{Leapfrog vs. velocity Verlet}

There is one main advantages for leapfrog: By starting the integration
at a different place, operators can be combined together explicitly
within a single step.

However, there are disadvantages for leapfrog as well.  It is
impossible to do perfect pressure coupling, since one never knows the
virial and the kinetic energy at the same time until AFTER the step is
over, which would require iteration to convergence.  For multi-step
processes, leapfrog is not explicitly reversible; in simple cases,
only little bit of simple algebra is required to generate a leapfrog
version of a Trotter factorization, but in more complicated cases,
it's not possible.

\subsection{Temperature coupling\index{temperature coupling}}

While direct use of molecular dynamics gives rise to the NVE ensemble,
most quantities that we wish to calculate are actually from the
canonical (NVT) ensemble. We propose in 5.0 to keep all of the
following thermostats: the {\em weak-coupling} scheme of
Berendsen~\cite{Berendsen84}, stochastic randomization through the
Andersen thermostat~\cite{Andersen80}, the extended ensemble
Nos{\'e}-Hoover scheme~\cite{Nose84,Hoover85}, or the
velocity-rescaling scheme of Bussi et al.~\cite{Bussi2007a} to
simulate constant temperature.  Of these, Berendsen does not actually
give the correct NVT ensemble; it allows one to enforce the correct
average kinetic energy, but the distribution of energies will be
incorrect.  We also propose to add stochastic Nose-Hoover methods,
where the bath variables have stochastic thermostats. We will remove
chain schemes for Nose-Hoover in 5.0. It's not clear how much
advantage they add, and they are more trouble than they are worth for
coding.

Another possibility for the Bussi thermostat is to thermostat each
atom, degree of freedom or constraint group{\em separately}. Because
of equipartition, each molecule will independently obey the correct
kinetic energy distribution.  This will require more random numbers
being generated, however, which may be expensive. The advantage is
that kinetic energies need not be summed to compute the thermostat. 

\subsubsection{Nos\'{e}-Hoover}
In the he extended-ensemble approach first proposed by
Nos{\'e}~\cite{Nose84} and later modified by Hoover~\cite{Hoover85},
the system Hamiltonian is extended by introducing a thermal reservoir
and a friction term in the equations of motion.  The friction force is
proportional to the product of each particle's velocity and a friction
parameter, $\xi$.  This friction parameter (or ``heat bath'' variable)
is a fully dynamic quantity with its own momentum ($p_{\xi}$) and
equation of motion; the time derivative is calculated from the
difference between the current kinetic energy and the reference
temperature.

In this formulation, the particles' equations of motion in
\figref{global} are replaced by:
\beq
\frac {\de^2\ve{r}_i}{\de t^2} = \frac{\ve{F}_i}{m_i} - 
\frac{p_{\xi}}{Q}\frac{\de \ve{r}_i}{\de t} ,
\label{eqn:NH-eqn-of-motion}
\eeq where the equation of motion for the heat bath parameter $\xi$ is:
\beq \frac {\de p_{\xi}}{\de t} = \left( T - T_0 \right).  \eeq The
reference temperature is denoted $T_0$, while $T$ is the current
instantaneous temperature of the system. The strength of the coupling
is determined by the constant $Q$ (usually called the ``mass parameter''
of the reservoir) in combination with the reference
temperature.~\footnote{Note that some derivations, an alternative
  notation $\xi_{\mathrm{alt}} = v_{\xi} = p_{\xi}/Q$ is used.}

The conserved quantity for the Nos{\'e}-Hoover equations of motion is not 
the total energy, but rather
\bea
H = \sum_{i=1}^{N} \frac{\pb_i}{2m_i} + U\left(\rv_1,\rv_2,\ldots,\rv_N\right) +\frac{p_{\xi}^2}{2Q} + N_fkT\xi,
\eea
where $N_f$ is the total number of degrees of freedom.

In our opinion, the mass parameter is a somewhat awkward way of
describing coupling strength, especially due to its dependence on
reference temperature (and some implementations even include the
number of degrees of freedom in your system when defining $Q$).  To
maintain the coupling strength, one would have to change $Q$ in
proportion to the change in reference temperature. For this reason, we
prefer to let the {\gromacs} user work instead with the period
$\tau_T$ of the oscillations of kinetic energy between the system and
the reservoir instead. It is directly related to $Q$ and $T_0$ via:
\beq
Q = \frac {\tau_T^2 T_0}{4 \pi^2}.
\eeq

It is however important to keep the difference between the
weak-coupling scheme and the Nos{\'e}-Hoover algorithm in mind: Using
weak coupling you get a strongly damped {\em exponential relaxation},
while the Nos{\'e}-Hoover approach produces an {\em oscillatory
  relaxation}.  The actual time it takes to relax with Nos{\'e}-Hoover
coupling is several times larger than the period of the oscillations
that you select. These oscillations (in contrast to exponential
relaxation) also means that the time constant normally should be 4--5
times larger than the relaxation time used with weak coupling, but
your mileage may vary.

Nos{\'e}-Hoover dynamics in simple systems such as collections of
harmonic oscillators, can be {\em nonergodic}, meaning that only a
subsection of phase space is ever sampled, even if the simulations
were to run for infinitely long.  For this reason, the Nos{\'e}-Hoover
chain approach was developed, where each of the Nos{\'e}-Hoover
thermostats has its own Nos{\'e}-Hoover thermostat controlling its
temperature.  In the limit of an infinite chain of thermostats, the
dynamics are guaranteed to be ergodic. Using just a few chains can
greatly improve the ergodicity, but recent research has shown that the
system will still be nonergodic, and it is still not entirely clear
what the practical effect of this~\cite{Cooke2008}. Currently, the
default number of chains is 10, but this can be controlled by the
user.  In the case of chains, the equations are modified in the
following way to include a chain of thermostatting
particles~\cite{Martyna1992}:

\bea
\frac {\de^2\ve{r}_i}{\de t^2} &~=~& \frac{\ve{F}_i}{m_i} - \frac{p_{{\xi}_1}}{Q_1} \frac{\de \ve{r}_i}{\de t} \nonumber \\
\frac {\de p_{{\xi}_1}}{\de t} &~=~& \left( T - T_0 \right) - p_{{\xi}_1} \frac{p_{{\xi}_2}}{Q_2} \nonumber \\
\frac {\de p_{{\xi}_{i=2\ldots N}}}{\de t} &~=~& \left(\frac{p_{\xi_{i-1}}^2}{Q_{i-1}} -kT\right) - p_{\xi_i} \frac{p_{\xi_{i+1}}}{Q_{i+1}} \nonumber \\
\frac {\de p_{\xi_N}}{\de t} &~=~& \left(\frac{p_{\xi_{N-1}}^2}{Q_{N-1}}-kT\right)
\label{eqn:NH-chain-eqn-of-motion}
\eea
The conserved quantity for Nos{\'e}-Hoover chains is
\bea
H = \sum_{i=1}^{N} \frac{\pb_i}{2m_i} + U\left(\rv_1,\rv_2,\ldots,\rv_N\right) +\sum_{k=1}^M\frac{p^2_{\xi_k}}{2Q^{\prime}_k} + N_fkT\xi_1 + kT\sum_{k=2}^M \xi_k 
\eea

The values and velocities of the Nos{\'e}-Hoover thermostat variables
are generally not included in the output, as they take up a fair
amount of space and are generally not important for analysis of
simulations, but this can be overridden by defining the environment
variable {\tt GMX\_NOSEHOOVER\_CHAINS}, which will print the values of
all the positions and velocities of all Nos{\'e}-Hoover particles in
the chain to the {\tt .edr} file.  Currently, leap-frog simulations
only have Nos{\'e}-Hoover chain lengths of 1, because of the
difficulty to adapting the Trotter algorithms to combining the
schemes.

In Gromacs 5.0, we propose to eliminate the Nos\'{e}-Hoover chains.
Although the chains add some chaos to the integrator, they are messy
to work with, and require storing and passing a large amount of
information that is unimportant for any physically relevant quantities.
It would be better to add chaos directly, through Langevin integration
of the bath variables (see Leimkuhler et al., J Stat Phys (2009) 135:
261–277, doi:10.1007/s10955-009-9734-0).
\bea
\frac {\de^2\ve{r}_i}{\de t^2} &~=~& \frac{\ve{F}_i}{m_i} - \frac{p_{{\xi}}}{Q} \frac{\de \ve{r}_i}{\de t} \nonumber \\
\frac {\de p_{{\xi}_1}}{\de t} &~=~& \left( T - T_0 \right) -\gamma \frac{p_{{\xi}}}{Q} + \sqrt{\frac{2\gamma}{\beta Q}}N(t) \nonumber \label{eqn:NH-Langevin}
\eea
Where $N(t)$ is a normally distributed random number, corresponding to a Weiner process.
This process can be Trotter split as:
\bea
iL = iL_1 + iL_2 + iL_{NH} + iL_{NH-stoc}
\eea
where: 
\bea
iL_1 &=& \sum_{i=1}^N \left[\frac{\pb_i}{m_i}\right]\cdot \frac{\partial}{\partial \rv_i} \nonumber \\
iL_2 &=& \sum_{i=1}^N \F_i\cdot \frac{\partial}{\partial \pb_i} \nonumber \\
iL_{\mathrm{NHC}} &=& \sum_{i=1}^N-\frac{p_{\xi}}{Q}\vv_i\cdot \nabla_{\vv_i} +\frac{p_{\xi}}{Q}\frac{\partial }{\partial \xi} + \left( T - T_0 \right)\frac{\partial }{\partial p_{\xi}} \\ 
iL_{\mathrm{NHC-stoc}} &=& -\frac{\gamma}{Q} \frac{\partial }{\partial p_{\xi}} + \sqrt{\frac{2\gamma}{\beta Q}} dW  
\eea
Where $dW$ is Weiner process.

(Add more information here).

As described in the integrator section, for temperature coupling, the
temperature that the algorithm attempts to match to the reference
temperature is calculated differently in velocity Verlet and leap-frog
dynamics.  Velocity Verlet ({\em md-vv}) uses the full-step kinetic
energy, while leap-frog and {\em md-vv-avek} use the
half-step-averaged kinetic energy.

We can examine the Trotter decomposition again to better understand
the differences between these constant-temperature integrators.  In
the case of Nos{\'e}-Hoover dynamics (for simplicity, using a chain
with $N=1$, with more details in Ref.~\cite{Martyna1996}), we split
the Liouville operator as
\beq
iL = iL_1 + iL_2 + iL_{\mathrm{NHC}},
\eeq
where
\bea
iL_1 &=& \sum_{i=1}^N \left[\frac{\pb_i}{m_i}\right]\cdot \frac{\partial}{\partial \rv_i} \nonumber \\
iL_2 &=& \sum_{i=1}^N \F_i\cdot \frac{\partial}{\partial \pb_i} \nonumber \\
iL_{\mathrm{NHC}} &=& \sum_{i=1}^N-\frac{p_{\xi}}{Q}\vv_i\cdot \nabla_{\vv_i} +\frac{p_{\xi}}{Q}\frac{\partial }{\partial \xi} + \left( T - T_0 \right)\frac{\partial }{\partial p_{\xi}}
\eea
For standard velocity Verlet with Nos{\'e}-Hoover temperature control, this becomes
\bea  
\exp(iL\dt) &=& \exp\left(iL_{\mathrm{NHC}}\dt/2\right) \exp\left(iL_2 \dt/2\right) \nonumber \\
&&\exp\left(iL_1 \dt\right) \exp\left(iL_2 \dt/2\right) \exp\left(iL_{\mathrm{NHC}}\dt/2\right) + \mathcal{O}(\Dt^3).
\eea
For half-step-averaged temperature control using {\em md-vv-avek},
this decomposition will not work, since we do not have the full step
temperature until after the second velocity step.  However, we can
construct an alternate decomposition that is still reversible, by
switching the place of the NHC and velocity portions of the
decomposition:
\bea  
\exp(iL\dt) &=& \exp\left(iL_2 \dt/2\right) \exp\left(iL_{\mathrm{NHC}}\dt/2\right)\exp\left(iL_1 \dt\right)\nonumber \\
&&\exp\left(iL_{\mathrm{NHC}}\dt/2\right) \exp\left(iL_2 \dt/2\right)+ \mathcal{O}(\Dt^3)
\label{eq:half_step_NHC_integrator}
\eea
This formalism allows us to easily see the difference between the
different flavors of velocity Verlet integrator.  The leap-frog
integrator can be seen as starting with
Eq.~\ref{eq:half_step_NHC_integrator} just before the $\exp\left(iL_1
\dt\right)$ term, yielding:
\bea  
\exp(iL\dt) &=&  \exp\left(iL_1 \dt\right) \exp\left(iL_{\mathrm{NHC}}\dt/2\right) \nonumber \\
&&\exp\left(iL_2 \dt\right) \exp\left(iL_{\mathrm{NHC}}\dt/2\right) + \mathcal{O}(\Dt^3)
\eea
and then using some algebra tricks to solve for some quantities are
required before they are actually calculated~\cite{Holian95}.

% }

\subsubsection{Berendsen temperature coupling\pawsindexquiet{Berendsen}{temperature coupling}\index{weak coupling}}
The Berendsen algorithm mimics weak coupling with first-order 
kinetics to an external heat bath with given temperature $T_0$. 
See ref.~\cite{Berendsen91} for a comparison with the
Nos{\'e}-Hoover scheme. The effect of this algorithm is
that a deviation of the system temperature from $T_0$ is slowly
corrected according to:
\beq
\frac{\de T}{\de t} = \frac{T_0-T}{\tau}
\label{eqn:Tcoupling}
\eeq
which means that a temperature deviation decays exponentially with a
time constant $\tau$.

The Berendsen thermostat suppresses the fluctuations of the kinetic
energy.  This means that one does not generate a proper canonical
ensemble, so rigorously, the sampling will be incorrect.  This
error scales with $1/N$, so for very large systems most ensemble
averages will not be affected significantly, except for the
distribution of the kinetic energy itself.  However, fluctuation
properties, such as the heat capacity, will be affected.  A similar
thermostat which does produce a correct ensemble is the velocity
rescaling thermostat~\cite{Bussi2007a} described below.

The heat flow into or out of the system is affected by scaling the
velocities of each particle every step, or every $n_\mathrm{TC}$ steps,
with a time-dependent factor $\lambda$, given by:
\beq 
\lambda = \left[ 1 + \frac{n_\mathrm{TC} \Delta t}{\tau_T}
\left\{\frac{T_0}{T(t -  \hDt)} - 1 \right\} \right]^{1/2}
\label{eqn:lambda}
\eeq
The parameter $\tau_T$ is close, but not exactly equal, to the time constant
$\tau$ of the temperature coupling (\eqnref{Tcoupling}):
\beq
\tau = 2 C_V \tau_T / N_{df} k
\eeq
where $C_V$ is the total heat capacity of the system, $k$ is Boltzmann's
constant, and $N_{df}$ is the total number of degrees of freedom. The
reason that $\tau \neq \tau_T$ is that the kinetic energy change
caused by scaling the velocities is partly redistributed between
kinetic and potential energy and hence the change in temperature is
less than the scaling energy.  In practice, the ratio $\tau / \tau_T$
ranges from 1 (gas) to 2 (harmonic solid) to 3 (water). When we use
the term ``temperature coupling time constant,'' we mean the parameter
\normindex{$\tau_T$}.  
{\bf Note} that in practice the scaling factor $\lambda$ is limited to 
the range of 0.8 $<= \lambda <=$ 1.25, to avoid scaling by very large
numbers which may crash the simulation. In normal use, 
$\lambda$ will always be much closer to 1.0.

\subsubsection{Velocity-rescaling temperature coupling\pawsindexquiet{velocity-rescaling}{temperature coupling}}
The velocity-rescaling thermostat~\cite{Bussi2007a} is essentially a Berendsen
thermostat (see above) with an additional stochastic term that ensures
a correct kinetic energy distribution by modifying it according to
\beq
\de K = (K_0 - K) \frac{\de t}{\tau_T} + 2 \sqrt{\frac{K K_0}{N_f}} \frac{\de W}{\sqrt{\tau_T}},
\label{eqn:vrescale}
\eeq
where $K$ is the kinetic energy, $N_f$ the number of degrees of freedom and $\de W$ a Wiener process.
There are no additional parameters, except for a random seed.
This thermostat produces a correct canonical ensemble and still has
the advantage of the Berendsen thermostat: first order decay of
temperature deviations and no oscillations. When an $NVT$ ensemble is used, the conserved energy quantity
is written to the energy and log file.  

\subsubsection{\normindex{Andersen thermostat}}
One simple way to maintain a thermostatted ensemble is to take an
$NVE$ integrator and periodically re-select the velocities of the
particles from a Maxwell-Boltzmann distribution.~\cite{Andersen80}.
This can either be done by randomizing all the velocities
simultaneously (massive collision) every $\tau_T/\Dt$ steps, or by
randomizing every particle with some small probability every timestep,
equal to $\Dt/\tau$, where in both cases $\Dt$ is the timestep and
$\tau_T$ is a characteristic coupling time scale.

Because of the way constraints operate, all particles in the same
constraint group must be re-randomized simultaneously.  This
thermostat is also only possible with velocity Verlet algorithms,
because it operates directly on the velocities at each timestep.

This algorithm avoids some of the ergodicity issues of other
algorithms, as energy cannot flow back and forth between energetically
decoupled components of the system as in velocity scaling motions.
However, it can slow down the kinetics of system by randomizing
correlated motions of the system, including slowing sampling when
$\tau_T$ is at moderate levels (less than 10 ps). This algorithm
should therefore generally not be used when examining kinetics of the
system, but can avoid ergodicity problems of scaling problems when
examining thermodynamic properties.

We will try to implement Andersen thermostat in the leapfrog case,
randomizing the $t + \hDt$ velocities.  This should be possible to do;
currently bookkeeping makes this complicated. 

Note that another advantage of the Andersen thermostat is that it
requires no calculation of the total kinetic energy, so that global
statistics are not limited by calculating this quantity.

% \ifthenelse{\equal{\gmxlite}{1}}{}{
\subsubsection{Nos{\'e}-Hoover temperature coupling\index{Nose-Hoover temperature coupling@Nos{\'e}-Hoover temperature coupling|see{temperature coupling, Nos{\'e}-Hoover}}{\index{temperature coupling Nose-Hoover@temperature coupling Nos{\'e}-Hoover}}\index{extended ensemble}}

The Berendsen weak-coupling algorithm is
extremely efficient for relaxing a system to the target temperature,
but once the system has reached equilibrium it might be more
important to probe a correct canonical ensemble. This is unfortunately
not the case for the weak-coupling scheme.

\subsubsection{Group temperature coupling}\index{temperature-coupling group}%
In {\gromacs} temperature coupling can be performed on groups of
atoms, typically a protein and solvent. The reason such algorithms
were introduced is that energy exchange between different components
is not perfect, due to different effects including cut-offs etc. If
now the whole system is coupled to one heat bath, water (which
experiences the largest cut-off noise) will tend to heat up and the
protein will cool down. Typically 100 K differences can be obtained.
With the use of proper electrostatic methods (PME) these difference
are much smaller but still not negligible.  The parameters for
temperature coupling in groups are given in the {\tt mdp} file.
Recent investigation has shown that small temperature differences
between protein and water may actually be an artifact of the way
temperature is calculated when there are finite timesteps, and very
large differences in temperature are likely a sign of something else
seriously going wrong with the system, and should be investigated
carefully~\cite{Eastwood2010}.

One special case should be mentioned: it is possible to T-couple only
part of the system, leaving other parts without temperature
coupling. This is done by specifying zero for the time constant
$\tau_T$ for the group of which should not be thermostatted.  If only
part of the system is thermostatted, the system will still eventually
converge to an NVT system.  In fact, one suggestion for minimizing
errors in the temperature caused by discretized timesteps is that if
constraints on the water are used, then only the water degrees of
freedom should be thermostatted, not protein degrees of freedom, as
the higher frequency modes in the protein can cause larger deviations
from the ``true'' temperature, the temperature obtained with small
timesteps~\cite{Eastwood2010}.

\subsection{Pressure coupling\index{pressure coupling}}
In the same spirit as the temperature coupling, the system can also be
coupled to a ``pressure bath.'' {\gromacs} supports both the Berendsen
algorithm~\cite{Berendsen84} that scales coordinates and box vectors
every step, the extended-ensemble Parrinello-Rahman approach~\cite{Parrinello81,Nose83}, and for
the velocity Verlet variants, the Martyna-Tuckerman-Tobias-Klein
(MTTK) implementation of pressure
control~\cite{Martyna1996}. Parrinello-Rahman and Berendsen can be
combined with any of the temperature coupling methods above; MTTK can
only be used with Nos{\'e}-Hoover temperature control.

\subsubsection{Berendsen pressure coupling\pawsindexquiet{Berendsen}{pressure coupling}\index{weak coupling}}
\label{sec:berendsen_pressure_coupling}
The Berendsen algorithm rescales the 
coordinates and box vectors every step, or every $n_\mathrm{PC}$ steps,
 with a matrix {\boldmath $\mu$},
which has the effect of a first-order kinetic relaxation of the pressure
towards a given reference pressure ${\bf P}_0$ according to
\beq
\frac{\de {\bf P}}{\de t} = \frac{{\bf P}_0-{\bf P}}{\tau_p}.
\eeq
The scaling matrix {\boldmath $\mu$} is given by
\beq
\mu_{ij}
= \delta_{ij} - \frac{n_\mathrm{PC}\Delta t}{3\, \tau_p} \beta_{ij} \{P_{0ij} - P_{ij}(t) \}.
\label{eqn:mu}
\eeq
\index{isothermal compressibility}
\index{compressibility}
Here, {\boldmath $\beta$} is the isothermal compressibility of the system.
In most cases this will be a diagonal matrix, with equal elements on the
diagonal, the value of which is generally not known.
It suffices to take a rough estimate because the value of {\boldmath $\beta$}
only influences the non-critical time constant of the
pressure relaxation without affecting the average pressure itself.
For water at 1 atm and 300 K 
$\beta = 4.6 \times 10^{-10}$ Pa$^{-1} = 4.6 \times 10^{-5}$ bar$^{-1}$,
which is $7.6 \times 10^{-4}$ MD units (see \chref{defunits}).
Most other liquids have similar values.
When scaling completely anisotropically, the system has to be rotated in
order to obey \eqnref{box_rot}.
This rotation is approximated in first order in the scaling, which is usually
less than $10^{-4}$. The actual scaling matrix {\boldmath $\mu'$} is
\beq
\mbox{\boldmath $\mu'$} = 
\left(\begin{array}{ccc}
\mu_{xx} & \mu_{xy} + \mu_{yx} & \mu_{xz} + \mu_{zx} \\
0        & \mu_{yy}            & \mu_{yz} + \mu_{zy} \\
0        & 0                   & \mu_{zz}
\end{array}\right).
\eeq
The velocities are neither scaled nor rotated.

In {\gromacs}, the Berendsen scaling can also be done isotropically, 
which means that instead of $\ve{P}$ a diagonal matrix with elements of size
trace$(\ve{P})/3$ is used. For systems with interfaces, semi-isotropic 
scaling can be useful.
In this case, the $x/y$-directions are scaled isotropically and the $z$
direction is scaled independently. The compressibility in the $x/y$ or
$z$-direction can be set to zero, to scale only in the other direction(s).

If you allow full anisotropic deformations and use constraints you
might have to scale more slowly or decrease your timestep to avoid
errors from the constraint algorithms.  It is important to note that
although the Berendsen pressure control algorithm yields a simulation
with the correct average pressure, it does not yield the exact NPT
ensemble, and it is not yet clear exactly what errors this approximation
may yield.

% \ifthenelse{\equal{\gmxlite}{1}}{}{
\subsubsection{Parrinello-Rahman pressure coupling\pawsindexquiet{Parrinello-Rahman}{pressure coupling}}

In cases where the fluctuations in pressure or volume are important
{\em per se} ({\eg} to calculate thermodynamic properties), especially
for small systems, it may be a problem that the exact ensemble is not
well defined for the weak-coupling scheme, and that it does not
simulate the true NPT ensemble.

{\gromacs} also supports constant-pressure simulations using the
Parrinello-Rahman approach~\cite{Parrinello81,Nose83}, which is similar
to the Nos{\'e}-Hoover temperature coupling, and in theory gives the
true NPT ensemble.  With the Parrinello-Rahman barostat, the box
vectors as represented by the matrix \ve{b} obey the matrix equation
of motion\footnote{The box matrix representation \ve{b} in {\gromacs}
corresponds to the transpose of the box matrix representation \ve{h}
in the paper by Nos{\'e} and Klein. Because of this, some of our
equations will look slightly different.}
\beq
\frac{\de \ve{b}^2}{\de t^2}= V \ve{W}^{-1} \ve{b}'^{-1} \left( \ve{P} - \ve{P}_{ref}\right).
\eeq

The volume of the box is denoted $V$, and $\ve{W}$ is a matrix parameter that determines
the strength of the coupling. The matrices \ve{P} and \ve{P}$_{ref}$ are the 
current and reference pressures, respectively.

The equations of motion for the particles are also changed, just as
for the Nos{\'e}-Hoover coupling. In most cases you would combine the 
Parrinello-Rahman barostat with the Nos{\'e}-Hoover
thermostat, but to keep it simple we only show the Parrinello-Rahman 
modification here:

\bea \frac {\de^2\ve{r}_i}{\de t^2} & = & \frac{\ve{F}_i}{m_i} -
\ve{M} \frac{\de \ve{r}_i}{\de t} , \\ \ve{M} & = & \ve{b}^{-1} \left[
  \ve{b} \frac{\de \ve{b}'}{\de t} + \frac{\de \ve{b}}{\de t} \ve{b}'
  \right] \ve{b}'^{-1}.  \eea The (inverse) mass parameter matrix
$\ve{W}^{-1}$ determines the strength of the coupling, and how the box
can be deformed.  The box restriction (\ref{eqn:box_rot}) will be
fulfilled automatically if the corresponding elements of $\ve{W}^{-1}$
are zero. Since the coupling strength also depends on the size of your
box, we prefer to calculate it automatically in {\gromacs}.  You only
have to provide the approximate isothermal compressibilities
{\boldmath $\beta$} and the pressure time constant $\tau_p$ in the
input file ($L$ is the largest box matrix element): \beq \left(
\ve{W}^{-1} \right)_{ij} = \frac{4 \pi^2 \beta_{ij}}{3 \tau_p^2 L}.
\eeq Just as for the Nos{\'e}-Hoover thermostat, you should realize
that the Parrinello-Rahman time constant is {\em not} equivalent to
the relaxation time used in the Berendsen pressure coupling algorithm.
In most cases you will need to use a 4--5 times larger time constant
with Parrinello-Rahman coupling. If your pressure is very far from
equilibrium, the Parrinello-Rahman coupling may result in very large
box oscillations that could even crash your run.  In that case you
would have to increase the time constant, or (better) use the weak-coupling
scheme to reach the target pressure, and then switch to
Parrinello-Rahman coupling once the system is in equilibrium.
Additionally, using the leap-frog algorithm, the pressure at time $t$
is not available until after the time step has completed, and so the
pressure from the previous step must be used, which makes the algorithm
not directly reversible, and may not be appropriate for high precision
thermodynamic calculations.

\subsubsection{Surface-tension coupling\pawsindexquiet{surface-tension}{pressure coupling}}
When a periodic system consists of more than one phase, separated by
surfaces which are parallel to the $xy$-plane,
the surface tension and the $z$-component of the pressure can be coupled
to a pressure bath. Presently, this only works with the Berendsen
pressure coupling algorithm in {\gromacs}.
The average surface tension $\gamma(t)$ can be calculated from
the difference between the normal and the lateral pressure
\bea
\gamma(t) & = & 
\frac{1}{n} \int_0^{L_z}
\left\{ P_{zz}(z,t) - \frac{P_{xx}(z,t) + P_{yy}(z,t)}{2} \right\} \mbox{d}z \\
& = &
\frac{L_z}{n} \left\{ P_{zz}(t) - \frac{P_{xx}(t) + P_{yy}(t)}{2} \right\},
\eea
where $L_z$ is the height of the box and $n$ is the number of surfaces.
The pressure in the z-direction is corrected by scaling the height of
the box with $\mu_z$
\beq
\Delta P_{zz} = \frac{\Delta t}{\tau_p} \{ P_{0zz} - P_{zz}(t) \}
\eeq
\beq
\mu_{zz} = 1 + \beta_{zz} \Delta P_{zz}
\eeq
This is similar to normal pressure coupling, except that the power
of $1/3$ is missing. 
The pressure correction in the $z$-direction is then used to get the
correct convergence for the surface tension to the reference value $\gamma_0$.
The correction factor for the box length in the $x$/$y$-direction is
\beq
\mu_{x/y} = 1 + \frac{\Delta t}{2\,\tau_p} \beta_{x/y}
        \left( \frac{n \gamma_0}{\mu_{zz} L_z}
        - \left\{ P_{zz}(t)+\Delta P_{zz} - \frac{P_{xx}(t) + P_{yy}(t)}{2} \right\} 
        \right)
\eeq
The value of $\beta_{zz}$ is more critical than with normal pressure
coupling. Normally an incorrect compressibility will just scale $\tau_p$,
but with surface tension coupling it affects the convergence of the surface
tension. 
When $\beta_{zz}$ is set to zero (constant box height), $\Delta P_z$ is also set
to zero, which is necessary for obtaining the correct surface tension. 

\subsubsection{MTTK pressure control algorithms}

As mentioned in the previous section, one weakness of leap-frog
integration is in constant pressure simulations, since the pressure
requires a calculation of both the virial and the kinetic energy at
the full time step; for leap-frog, this information is not available
until {\em after} the full timestep.  Velocity Verlet does allow the
calculation, at the cost of an extra round of global communication,
and can compute, mod any integration errors, the true NPT ensemble.

The full equations, combining both pressure coupling and temperature
coupling, are taken from Martyna {\em et al.}~\cite{Martyna1996} and
Tuckerman~\cite{Tuckerman2006} and are referred to here as MTTK
equations (Martyna-Tuckerman-Tobias-Klein).  We introduce for
convenience $\epsilon = (1/3)\ln (V/V_0)$, where $V_0$ is a reference
volume.  The momentum of $\epsilon$ is $\veps = p_{\epsilon}/W =
\dot{\epsilon} = \dot{V}/3V$, and define $\alpha = 1 + 3/N_{dof}$ (see
Ref~\cite{Tuckerman2006})

Nonisotropic equations have been derived by Tuckerman.  Below are the
isobaric equations.  We will make the thermostat steps more general
(i.e. other thermostats other than NH allowed here), as well as
controlling the barostat temperature by Langevin equation.  This
derivation will need to be completed.

The isobaric equations are
\bea
\dot{\rv}_i &=& \frac{\pb_i}{m_i} + \frac{\peps}{W} \rv_i \nonumber \\
\frac{\dot{\pb}_i}{m_i} &=& \frac{1}{m_i}\F_i - \alpha\frac{\peps}{W} \frac{\pb_i}{m_i} \nonumber \\
\dot{\epsilon} &=& \frac{\peps}{W} \nonumber \\
\frac{\dot{\peps}}{W} &=& \frac{3V}{W}(P_{\mathrm{int}} - P) + (\alpha-1)\left(\sum_{n=1}^N\frac{\pb_i^2}{m_i}\right),\\
\eea
where
\bea
P_{\mathrm{int}} &=& P_{\mathrm{kin}} -P_{\mathrm{vir}} = \frac{1}{3V}\left[\sum_{i=1}^N \left(\frac{\pb_i^2}{2m_i} - \rv_i \cdot \F_i\
\right)\right].
\eea
The terms including $\alpha$ are required to make phase space
incompressible~\cite{Tuckerman2006}. The $\epsilon$ acceleration term
can be rewritten as
\bea
\frac{\dot{\peps}}{W} &=& \frac{3V}{W}\left(\alpha P_{\mathrm{kin}} - P_{\mathrm{vir}} - P\right)
\eea
In terms of velocities, these equations become
\bea
\dot{\rv}_i &=& \vv_i + \veps \rv_i \nonumber \\
\dot{\vv}_i &=& \frac{1}{m_i}\F_i - \alpha\veps \vv_i \nonumber \\
\dot{\epsilon} &=& \veps \nonumber \\
\dot{\veps} &=& \frac{3V}{W}(P_{\mathrm{int}} - P) + (\alpha-1)\left( \sum_{n=1}^N \frac{1}{2} m_i \vv_i^2\right)\nonumber \\
P_{\mathrm{int}} &=& P_{\mathrm{kin}} - P_{\mathrm{vir}} = \frac{1}{3V}\left[\sum_{i=1}^N \left(\frac{1}{2} m_i\vv_i^2 - \rv_i \cdot \F_i\right)\right]
\eea
For these equations, the conserved quantity is
\bea
H = \sum_{i=1}^{N} \frac{\pb_i^2}{2m_i} + U\left(\rv_1,\rv_2,\ldots,\rv_N\right) + \frac{p_\epsilon}{2W} + PV
\eea
The next step is to add temperature control.  Adding Nos{\'e}-Hoover
chains, including to the barostat degree of freedom, where we use
$\eta$ for the barostat Nos{\'e}-Hoover variables, and $Q^{\prime}$
for the coupling constants of the thermostats of the barostats, we get
\bea
\dot{\rv}_i &=& \frac{\pb_i}{m_i} + \frac{\peps}{W} \rv_i \nonumber \\
\frac{\dot{\pb}_i}{m_i} &=& \frac{1}{m_i}\F_i - \alpha\frac{\peps}{W} \frac{\pb_i}{m_i} - \frac{p_{\xi_1}}{Q_1}\frac{\pb_i}{m_i}\nonumber \\
\dot{\epsilon} &=& \frac{\peps}{W} \nonumber \\
\frac{\dot{\peps}}{W} &=& \frac{3V}{W}(\alpha P_{\mathrm{kin}} - P_{\mathrm{vir}} - P) -\frac{p_{\eta_1}}{Q^{\prime}_1}\peps \nonumber \\
\dot{\xi}_k &=& \frac{p_{\xi_k}}{Q_k} \nonumber \\ 
\dot{\eta}_k &=& \frac{p_{\eta_k}}{Q^{\prime}_k} \nonumber \\
\dot{p}_{\xi_k} &=& G_k - \frac{p_{\xi_{k+1}}}{Q_{k+1}} \;\;\;\; k=1,\ldots, M-1 \nonumber \\ 
\dot{p}_{\eta_k} &=& G^\prime_k - \frac{p_{\eta_{k+1}}}{Q^\prime_{k+1}} \;\;\;\; k=1,\ldots, M-1 \nonumber \\
\dot{p}_{\xi_M} &=& G_M \nonumber \\
\dot{p}_{\eta_M} &=& G^\prime_M, \nonumber \\
\eea
where
\bea
P_{\mathrm{int}} &=& P_{\mathrm{kin}} - P_{\mathrm{vir}} = \frac{1}{3V}\left[\sum_{i=1}^N \left(\frac{\pb_i^2}{2m_i} - \rv_i \cdot \F_i\right)\right] \nonumber \\
G_1  &=& \sum_{i=1}^N \frac{\pb^2_i}{m_i} - N_f kT \nonumber \\
G_k  &=&  \frac{p^2_{\xi_{k-1}}}{2Q_{k-1}} - kT \;\; k = 2,\ldots,M \nonumber \\
G^\prime_1 &=& \frac{\peps^2}{2W} - kT \nonumber \\
G^\prime_k &=& \frac{p^2_{\eta_{k-1}}}{2Q^\prime_{k-1}} - kT \;\; k = 2,\ldots,M
\eea
The conserved quantity is now
\bea
H = \sum_{i=1}^{N} \frac{\pb_i}{2m_i} + U\left(\rv_1,\rv_2,\ldots,\rv_N\right) + \frac{p^2_\epsilon}{2W} + PV + \nonumber \\
\sum_{k=1}^M\frac{p^2_{\xi_k}}{2Q_k} +\sum_{k=1}^M\frac{p^2_{\eta_k}}{2Q^{\prime}_k} + N_fkT\xi_1 +  kT\sum_{i=2}^M \xi_k + kT\sum_{k=1}^M \eta_k
\eea
Returning to the Trotter decomposition formalism, for pressure control and temperature control~\cite{Martyna1996} we get:
\bea
iL = iL_1 + iL_2 + iL_{\epsilon,1} + iL_{\epsilon,2} + iL_{\mathrm{NHC-baro}} + iL_{\mathrm{NHC}}
\eea
where ``NHC-baro'' corresponds to the Nos{\`e}-Hoover chain of the barostat,
and NHC corresponds to the NHC of the particles,
\bea
iL_1 &=& \sum_{i=1}^N \left[\frac{\pb_i}{m_i} + \frac{\peps}{W}\rv_i\right]\cdot \frac{\partial}{\partial \rv_i} \\
iL_2 &=& \sum_{i=1}^N \F_i - \alpha \frac{\peps}{W}\pb_i \cdot \frac{\partial}{\partial \pb_i} \\
iL_{\epsilon,1} &=& \frac{p_\epsilon}{W} \frac{\partial}{\partial \epsilon}\\
iL_{\epsilon,2} &=& G_{\epsilon} \frac{\partial}{\partial p_\epsilon}
\eea
and where
\bea
G_{\epsilon} = 3V\left(\alpha P_{\mathrm{kin}} - P_{\mathrm{vir}} - P\right)
\eea 
Using the Trotter decomposition, we get
\bea  
\exp(iL\dt) &=& \exp\left(iL_{\mathrm{NHC-baro}}\dt/2\right)\exp\left(iL_{\mathrm{NHC}}\dt/2\right) \nonumber \nonumber \\
&&\exp\left(iL_{\epsilon,2}\dt/2\right) \exp\left(iL_2 \dt/2\right) \nonumber \nonumber \\
&&\exp\left(iL_{\epsilon,1}\dt\right) \exp\left(iL_1 \dt\right) \nonumber \nonumber \\
&&\exp\left(iL_2 \dt/2\right) \exp\left(iL_{\epsilon,2}\dt/2\right) \nonumber \nonumber \\
&&\exp\left(iL_{\mathrm{NHC}}\dt/2\right)\exp\left(iL_{\mathrm{NHC-baro}}\dt/2\right) + \mathcal{O}(\dt^3)
\eea
The action of $\exp\left(iL_1 \dt\right)$ comes from the solution of
the the differential equation 
$\dot{\rv}_i = \vv_i + \veps \rv_i$
with $\vv_i = \pb_i/m_i$ and $\veps$ constant with initial condition
$\rv_i(0)$, evaluate at $t=\Delta t$.  This yields the evolution
\beq
\rv_i(\dt) = \rv_i(0)e^{\veps \dt} + \Delta t \vv_i(0) e^{\veps \dt/2} \sinhx{\veps \dt/2}.
\eeq
The action of $\exp\left(iL_2 \dt/2\right)$ comes from the solution
of the differential equation $\dot{\vv}_i = \frac{\F_i}{m_i} -
\alpha\veps\vv_i$, yielding
\beq
\vv_i(\dt/2) = \vv_i(0)e^{-\alpha\veps \dt/2} + \frac{\Delta t}{2m_i}\F_i(0) e^{-\alpha\veps \dt/4}\sinhx{\alpha\veps \dt/4}.
\eeq
{\em md-vv-avek} uses the full step kinetic energies for determining the pressure with the pressure control,
but the half-step-averaged kinetic energy for the temperatures, which can be written as a Trotter decomposition as
\bea  
\exp(iL\dt) &=& \exp\left(iL_{\mathrm{NHC-baro}}\dt/2\right)\nonumber \exp\left(iL_{\epsilon,2}\dt/2\right) \exp\left(iL_2 \dt/2\right) \nonumber \\
&&\exp\left(iL_{\mathrm{NHC}}\dt/2\right) \exp\left(iL_{\epsilon,1}\dt\right) \exp\left(iL_1 \dt\right) \exp\left(iL_{\mathrm{NHC}}\dt/2\right) \nonumber \\
&&\exp\left(iL_2 \dt/2\right) \exp\left(iL_{\epsilon,2}\dt/2\right) \exp\left(iL_{\mathrm{NHC-baro}}\dt/2\right) + \mathcal{O}(\dt^3)
\eea
With constraints, the equations become significantly more
complicated, in that each of these equations need to be solved
iteratively for the constraint forces.  The discussion of the details of the iteration
is beyond the scope of this manual; readers are encouraged to see the
implementation described in~\cite{Yu2010}.

\subsection{Monte Carlo Pressure control}

All of these pressure control algorithms are a huge mess.  Another
option is to do Monte Carlo Barostat control.  In this case, we
generate a move in the volume distribution.

The simplest way to do this is scaling all lengths by the same amount.

Another way is to scale the distances between all molecules, while not
scaling the distances 

\subsection{Multistep integrators}

\subsubsection{Infrequent evaluation of temperature and pressure coupling}

Temperature and pressure control require global communication to
compute the kinetic energy and virial, which can become costly if
performed every step for large systems.  We can rearrange the Trotter
decomposition to give alternate symplectic, reversible integrator with
the coupling steps every $n$ steps instead of every steps.  These new
integrators will diverge if the coupling time step is too large, as
the auxiliary variable integrations will not converge.  However, in
most cases, long coupling times are more appropriate, as they disturb
the dynamics less~\cite{Martyna1996}.

Standard velocity Verlet with Nos{\'e}-Hoover temperature control has a Trotter expansion
\bea  
\exp(iL\dt) &\approx& \exp\left(iL_{\mathrm{NHC}}\dt/2\right) \exp\left(iL_2 \dt/2\right) \nonumber \\
&&\exp\left(iL_1 \dt\right) \exp\left(iL_2 \dt/2\right) \exp\left(iL_{\mathrm{NHC}}\dt/2\right).
\eea
If the Nos{\'e}-Hoover chain is sufficiently slow with respect to the motions of the system, we can
write an alternate integrator over $n$ steps for velocity Verlet as~\cite{reversible}
\bea  
\exp(iL\dt) &\approx& \exp\left(iL_{\mathrm{NHC}}(n\dt/2)\right)\left[\exp\left(iL_2 \dt/2\right)\right. \nonumber \\
            &        &\left.\exp\left(iL_1 \dt\right) \exp\left(iL_2 \dt/2\right)\right]^n \exp\left(iL_{\mathrm{NHC}}(n\dt/2)\right)
\eea

For leapfrog Verlet, we write:
\bea  
\exp(inL\dt) &\approx& \left[\exp\left(iL_2 \dt/2\right)\exp\left(iL_1 \dt\right) \exp\left(iL_2 \dt/2\right)\right]^n \exp\left(iL_{\mathrm{NHC}}(n\dt)\right) \nonumber \\
   &\approx& \exp\left(iL_2 \dt/2\right) \left[\exp\left(iL_1 \dt\right) \exp\left(iL_2 \dt\right)\right]^n \exp\left(-iL_2 \dt/2\right)\exp\left(iL_{\mathrm{NHC}}(n\dt)\right)\nonumber \\
   &\approx& \exp\left(-iL_2 \dt/2\right)\exp\left(iL_{\mathrm{NHC}}(n\dt)\right)\exp\left(iL_2 \dt/2\right) \left[\exp\left(iL_1 \dt\right) \exp\left(iL_2 \dt\right)\right]^n\nonumber
\eea

We can carry out the last step is because we don't actually care what
the initial state of the Nose-Hoover variables is.  We are left with
the problem that the Nose-Hoover step should be acting on the full
step velocities. Currently (4.6) in GROMACS infrequent temperature
control is implemented as:
\bea  
\exp(inL\dt) &=& [\exp\left(iL_1 \dt\right)\exp\left(iL_2 \dt\right)]^n \exp\left(iL_{\mathrm{NHC}}n\dt\right)
\eea
This makes the approximation that $L_{\mathrm{NHC}}$ commutes with
$L_2$, which does not quite seem right. However, it seems to be
working OK.  Whether this is an irrelevant approximation or works out because of 
nifty algebra tricks should be investigated more completely.

For pressure control, this becomes
\bea  
\exp(iL\dt) &\approx& \exp\left(iL_{\mathrm{NHC-baro}}(n\dt/2)\right)\exp\left(iL_{\mathrm{NHC}}(n\dt/2)\right) \nonumber \nonumber \\
&&\exp\left(iL_{\epsilon,2}(n\dt/2)\right) \left[\exp\left(iL_2 \dt/2\right)\right. \nonumber \nonumber \\
&&\exp\left(iL_{\epsilon,1}\dt\right) \exp\left(iL_1 \dt\right) \nonumber \nonumber \\
&&\left.\exp\left(iL_2 \dt/2\right)\right]^n \exp\left(iL_{\epsilon,2}(n\dt/2)\right) \nonumber \nonumber \\
&&\exp\left(iL_{\mathrm{NHC}}(n\dt/2)\right)\exp\left(iL_{\mathrm{NHC-baro}}(n\dt/2)\right),
\eea
where the box {\em volume} integration occurs every step, but the auxiliary variable
integrations happen every $n$ steps.

For leapfrog, we can write this as:
\bea  
\exp(iL\dt) &\approx& \left[\exp\left(iL_{\epsilon,1}\dt\right) \exp\left(iL_1 \dt\right) \exp\left(iL_2 \dt\right)\right]^n \nonumber \\
&& \exp\left(iL_{\epsilon,2}(n\dt/2)\right) \exp\left(iL_{\mathrm{NHC}}(n\dt/2)\right)\exp\left(iL_{\mathrm{NHC-baro}}(n\dt)\right) \nonumber \\
&& \exp\left(iL_{\mathrm{NHC}}(n\dt/2)\right) \exp\left(iL_{\epsilon,2}(n\dt/2)\right) \nonumber
\nonumber
\eea

Currently, in 4.6, the MTTK scheme is not implemented. However, it is implemented in Parrinello-Rahman (without a thermostat for the barostat degree of freedom) as:
\bea  
\exp(iL\dt) &\approx& \left[\exp\left(iL_{\epsilon,1}\dt\right) \exp\left(iL_1 \dt\right)\exp\left(iL_2 \dt\right)\right]^n \nonumber \\
&&\exp\left(iL_{\epsilon,2}(n\dt)\right) \exp\left(iL_{\mathrm{NHC}}(n\dt)\right)\nonumber
\eea
This appears to make the assumption that $L_{\mathrm{NHC}}$ commutes
with $L_{\epsilon,2}$. More thinking needs to be done if this is a
reasonable assumption.  $L_{\epsilon,2}$ involves the pressure, which
may change slightly as the kinetic energy changes.

\subsubsection{Energy decomposition}

Various groups have found that one cannot run the Fourier space PME
contribution any less than every 2 or 3 times less frequently than the
real space contribution, since the Fourier space contribution has some
short time components.

However, one could construct an explicitly short range / long range
decomposition.  Let's carry out the short range electrostatics with
some type of cutoff dynamics, most likely a shifted cutoff, that
explicitly ignores long range charges.

\bea
E_{\mathrm{short}} &=& E_{\mathrm{cutoff}} \\ 
E_{\mathrm{long}} &=& E_{\mathrm{Fourier}} + E_{\mathrm{realspace}} - E_{\mathrm{cutoff}}
\eea

This decomposition would have correct long time step energies, while
at the same time allowing the inner step to be evaluated less
frequently than if a PME Fourier/real space decomposition was
performed.

The Liouville operator is split into three components:

\beq
iL = iL_p + iL_{FSR} + iL_{FLR}
\eeq
where
\bea 
iL_{p} &=& \sum_{i=1}^N \left[\frac{\pb_i}{m_i}\right]\cdot \frac{\partial}{\partial \rv_i} \nonumber \\
iL_{FSR} &=& \sum_{i=1}^N \F_{SR,i}\cdot \frac{\partial}{\partial \pb_i} \nonumber \\
iL_{FLR} &=& \sum_{i=1}^N \F_{LR,i}\cdot \frac{\partial}{\partial \pb_i} \nonumber
\eea
This becomes
\bea 
\exp(iL\Dt) &=& \exp(iL_{FLR}\hDt) \exp(iL_{FSR}\hDt) \exp(iL_p\Dt) \exp(iL_{FSR}\hDt) \exp(iL_{FLR}\hDt) \nonumber \\
            &=& \exp(iL_{FLR}\hDt) [\exp(iL_{FSR}\frac{\Dt}{2n}) \exp(iL_p\frac{\Dt}{n}) \exp(iL_{FSR}\frac{\Dt}{2n})]^n \exp(iL_{FLR}\hDt) \nonumber 
\eea

Alternatively, by shifting, a single step becomes:
\bea
\exp(iL\Dt) &=& [\exp(iL_{FSR}\frac{\Dt}{2n}) \exp(iL_p\frac{\Dt}{n}) \exp(iL_{FSR}\frac{\Dt}{2n})]^n \exp(iL_{FLR}\Dt) \nonumber \\
            &=& \exp(iL_{FSR}\frac{\Dt}{2n}) [\exp(iL_p\frac{\Dt}{n}) \exp(iL_{FSR}\frac{\Dt}{n})]^n \exp(-iL_{FSR}\frac{\Dt}{2n}) \exp(iL_{FLR}\Dt) \nonumber \\
            &=& [\exp(iL_p\frac{\Dt}{n}) \exp(iL_{FSR}\frac{\Dt}{n})]^n \exp(-iL_{FSR}\frac{\Dt}{2n}) \exp(iL_{FLR}\Dt) \exp(iL_{FSR}\frac{\Dt}{2n})\nonumber \\
            &=& [\exp(iL_p\frac{\Dt}{n}) \exp(iL_{FSR}\frac{\Dt}{n})]^n \exp(iL_{FLR}\Dt)\nonumber
\eea
This last step is possible because $L_{FLR}$ and $L_{SLR}$ commute; 
it does not matter what order we add the components to the velocity. 

This can be seen as the a leap-frog version of the reversible Trotter
decomposition scheme~\cite{Tuckerman1992a} currently used. The long-range forces are determined every $n$ steps and
are then integrated using a
time step of $\Dt_\mathrm{LR} = n \Dt$: \beq \ve{v}(t+\hDt) =
\left\{ \begin{array}{lll} \displaystyle \ve{v}(t-\hDt) +
  \frac{1}{m}\left[\ve{F}_\mathrm{SR}(t) + n
    \ve{F}_\mathrm{LR}(t)\right] \Dt &,& \mathrm{step} ~\%~ n = 0
  \\ \noalign{\medskip} \displaystyle \ve{v}(t-\hDt) +
  \frac{1}{m}\ve{F}_\mathrm{SR}(t)\Dt &,& \mathrm{step} ~\%~ n \neq 0
\end{array} \right.
\eeq

The parameter $n$ is equal to the neighbor list update frequency.

In the case of three stages of potential energy, we get: 
\[iL = iL_p + iL_{FVSR} + iL_{FSR} + iL_{FLR} \]
and:
\bea 
\exp\left(iL\Dt\right) &=& \exp\left(iL_{FLR}\hDt\right) \left[\exp\left(iL_{FSR}\frac{\Dt}{2n}\right) \right.\nonumber \\
        && \left[\exp\left(iL_{FVSR}\frac{\Dt}{2nm}\right)\exp\left(iL_v\frac{\Dt}{nm}\right)\exp\left(iL_{FVSR}\frac{\Dt}{2nm}\right)\right]^m \nonumber \\
        &&\left.\exp\left(iL_{FSR}\frac{\Dt}{2n}\right)\right]^n \exp\left(iL_{FLR}\hDt\right) \nonumber
\eea

Because of the commutivity of the force operators, this can be rewritten as:
\bea 
\exp\left(iL\Dt\right) &=& \left[\left[\exp\left(iL_p\frac{\Dt}{nm}\right)\exp\left(iL_{FVSR}\frac{\Dt}{nm}\right)\right]^m\right.
\nonumber \\        
&&\left.\exp\left(iL_{FSR}\frac{\Dt}{n}\right)\right]^n \exp\left(iL_{FLR}\hDt\right) \nonumber
\eea

This can be continued indefinitely as required.

To save computation time, slowly varying forces can be calculated less
often than rapidly varying forces. In {\gromacs} such a
\normindex{multiple time step} splitting is possible between short and
long range non-bonded interactions.  In {\gromacs} versions up to 4.0,
an irreversible integration scheme was used which is also used by the
{\gromos} simulation package: every $n$ steps the long range forces
are determined and these are then also used (without modification) for
the next $n-1$ integration steps in \eqnref{leapfrogv}. Such an
irreversible scheme can result in bad energy conservation and,
possibly, bad sampling.

This is equivalent to:

\subsubsection{Avoiding constraints in interactions}

Constraints make integration complicated, especially pressure control.
There is no rule that there need only be one level of energy
decomposition; one can run multiple levels of decomposition together.

\beq
E_{\mathrm{total}} = E_{\mathrm{bonded}} + E_{\mathrm{short}} + E_{\mathrm{long}}
\eeq

The bonded terms could be evaluated every 0.5 fs, short range terms
every 2 fs, and long range terms every 10-20 fs, depending on how the
long range terms are written.

\begin{comment}
\section{Code organization}

{\bf Starting mdrun}

Begin with system setup.  At the end of system setup, we have the
state objects defined. All parameters are assigned, all particles with
initial conditions, and all particles divided between cores.  

{\bf Main loop}

\begin{enumerate}
\item Decide if MC or MD.  Can be either random (according to some
  fixed percentage), or fixed numbers of each type of move. Global update of neighborlist
\item If MC, calculate energy for the whole system, If MD, calculate
  energies forces for the whole system. Global update of neighborlist.
\item Increment the system (MD or MC: see below).  May contain force calculations of components.
\item Decide to accept or reject the new coordinates (if pure MD, acceptance) 
\item Perform sampling in state space (replica exchange).
\end{enumerate}

{\bf MD loop}
\begin{enumerate}
\item xxx
\end{enumerate}


{\bf MC loop}
\begin{enumerate} 
\item Make position move
\item perform multistage MC  
\item resample kinetic energies  
\end{enumerate}
\end{comment}
%{\bf Example: NVT MD with leapfrog and infrequent temperature coupling}
%
%for velocity Verlet:
%\begin{itemize}
%\item increment Nose-Hoover variables by $\Dt$
%\end{itemize}
%{\bf Example: NPT MD using  MC barostat}
%
%{\bf Example: Hybrid MC/MD}
%
%{\bf Example:  }


% } % Brace matches ifthenelse test for gmxlite


%%\subsection{The complete update algorithm}
%\begin{figure}
%\begin{center}
%\addtolength{\fboxsep}{0.5cm}
%\begin{shadowenv}[12cm]
%{\large \bf THE UPDATE ALGORITHM}
%\rule{\textwidth}{2pt} \\
%Given:\\
%Positions $\ve{r}$ of all atoms at time $t$ \\
%Velocities $\ve{v}$ of all atoms at time $t-\hDt$ \\
%Accelerations $\ve{F}/m$ on all atoms at time $t$.\\
%(Forces are computed disregarding any constraints)\\
%Total kinetic energy and virial at $t-\Dt$\\
%$\Downarrow$ \\
%{\bf 1.} Compute the scaling factors $\lambda$ and $\mu$\\
%according to \eqnsref{lambda}{mu}\\   
%$\Downarrow$ \\
%{\bf 2.} Update and scale velocities: $\ve{v}' =  \lambda (\ve{v} +
%\ve{a} \Delta t)$ \\
%$\Downarrow$ \\
%{\bf 3.} Compute new unconstrained coordinates: $\ve{r}' = \ve{r} + \ve{v}'
%\Delta t$ \\
%$\Downarrow$ \\
%{\bf 4.} Apply constraint algorithm to coordinates: constrain($\ve{r}^{'} \rightarrow  \ve{r}'';
%\,  \ve{r}$) \\
%$\Downarrow$ \\
%{\bf 5.} Correct velocities for constraints: $\ve{v} = (\ve{r}'' -
%\ve{r}) / \Delta t$ \\
%$\Downarrow$ \\
%{\bf 6.} Scale coordinates and box: $\ve{r} = \mu \ve{r}''; \ve{b} =
%\mu  \ve{b}$ \\
%\end{shadowenv}
%\caption{The MD update algorithm with the leap-frog integrator}
%\label{fig:complete-update}
%\end{center}
%\end{figure}
%The complete algorithm for the update of velocities and coordinates is
%given using leap-frog in \figref{complete-update}. The SHAKE algorithm of step
%4 is explained below. 

%{\gromacs} has a provision to ``freeze''  (prevent motion of) selected
%particles\index{frozen atoms}, which must be defined as a ``\swapindex{freeze}{group}.'' This is implemented
%using a {\em freeze factor $\ve{f}_g$}, which is a vector, and differs for each
%freeze group (see \secref{groupconcept}). This vector contains only
%zero (freeze) or one (don't freeze).
%When we take this freeze factor and the external acceleration $\ve{a}_h$ into 
%account the update algorithm for the velocities becomes
%\beq
%\ve{v}(t+\hdt)~=~\ve{f}_g * \lambda * \left[ \ve{v}(t-\hdt) +\frac{\ve{F}(t)}{m}\Delta t + \ve{a}_h \Delta t \right],
%\eeq
%where $g$ and $h$ are group indices which differ per atom.

\subsection{Output step}
The most important output of the MD run is the {\em
\swapindex{trajectory}{file}}, which contains particle coordinates
and (optionally) velocities at regular intervals.
The trajectory file contains frames that could include positions,
velocities and/or forces, as well as information about the dimensions
of the simulation volume, integration step, integration time, etc. The
interpretation of the time varies with the integrator chosen, as
described above. For velocity-Verlet integrators, velocities labeled
at time $t$ are for that time. For other integrators (e.g. leap-frog,
stochastic dynamics), the velocities labeled at time $t$ are for time
$t - \hDt$.

%Since the trajectory files are lengthy, one should not save every
%step! To retain all information it suffices to write a frame every 15
%steps, since at least 30 steps are made per period of the highest
%frequency in the system, and Shannon's \normindex{sampling} theorem
%states that two samples per period of the highest frequency in a
%band-limited signal contain all available information. But that still
%gives very long files! So, if the highest frequencies are not of
%interest, 10 or 20 samples per ps may suffice. Be aware of the
%distortion of high-frequency motions by the {\em stroboscopic effect},
%called {\em aliasing}: higher frequencies are mirrored with respect to
%the sampling frequency and appear as lower frequencies.

%{\gromacs} can also write reduced-precision coordinates for a subset of
%the simulation system to a special compressed trajectory file
%format. All the other tools can read and write this format. See
%\secref{mdpopt} for details on how to set up your {\tt .mdp} file
%to have {\tt mdrun} use this feature.

\section{Constraint algorithms\index{constraint algorithms}}

Constraints will still be possible to imposed in {\gromacs} using
LINCS (default) or the traditional SHAKE method.  With the new
addition of subtraction off of the velocity components for both
methods, either method will be possible to plug in for any integrator.
However, there are real problems with barostats, addressed earlier.

% } % Brace matches ifthenelse test for gmxlite

\newcommand{\vrond}{\stackrel{\circ}{\ve{r}}}
\newcommand{\rond}{\stackrel{\circ}{r}}
\newcommand{\ruis}{\ve{r}^G}

% \ifthenelse{\equal{\gmxlite}{1}}{}{
\section{Stochastic Dynamics\swapindexquiet{stochastic}{dynamics}}
\label{sec:SD}
Stochastic or velocity \swapindex{Langevin}{dynamics} adds a friction
and a noise term to Newton's equations of motion, as
\beq
\label{SDeq}
m_i {\de^2 \ve{r}_i \over \de t^2} =
- m_i \xi_i {\de \ve{r}_i \over \de t} + \ve{F}_i(\ve{r}) + \vrond_i,
\eeq 
where $\xi_i$ is the friction constant $[1/\mbox{ps}]$ and
$\vrond_i\!\!(t)$  is a noise process with 
$\langle \rond_i\!\!(t) \rond_j\!\!(t+s) \rangle = 
    2 m_i \xi_i k_B T \delta(s) \delta_{ij}$.
When $1/\xi_i$ is large compared to the time scales present in the system,
one could see stochastic dynamics as molecular dynamics with stochastic
temperature-coupling. The advantage compared to MD with Berendsen
temperature-coupling is that in case of SD the generated ensemble is known.
For simulating a system in vacuum there is the additional advantage that there is no
accumulation of errors for the overall translational and rotational
degrees of freedom.
When $1/\xi_i$ is small compared to the time scales present in the system,
the dynamics will be completely different from MD, but the sampling is
still correct.

In {\gromacs} there are two algorithms to integrate equation
(\ref{SDeq}): a simple and efficient one and a more complex leap-frog
algorithm~\cite{Gunsteren88}.  The accuracy of both integrators is
equivalent to the normal MD leap-frog and velocity-Verlet integrator,
except with constraints where the simple SD integrator is
significantly less accurate.

We propose to use the Sivak, Chodera, and Crooks
(http://arxiv.org/pdf/1301.3800v2.pdf) OVRVO decomposition for
stochastic dynamics.  The formulation is explicitly in Trotter
factorization form.  We note that this integrator allows one to
implement a number of different integrators -- Bussi et al., Brownian
dynamics, and standard stochastic integration.

\subsubsection{OVRVO process of Sivak, Chodera, and Crooks}

The full SCC integrator (including Hamiltonian updates) is:
\bea
v(t + \Dt/4) &=& \sqrt{a}v(t) + \sqrt{\frac{1-a}{\beta m}} \mathcal{N}^{+}(t) \\
v(t + \Dt/2) &=& v(t + \Dt/4) + \frac{b\Dt}{2m} f(t) \\
x(t + \Dt/2) &=& x(t) + \frac{b\Dt}{2} v(t + \Dt/2) \\
\mathcal{H}(t+\Delta t) &=& \Lambda(t) \\
x(t + \Dt) &=& x(t + \Dt/2) + \frac{b\Dt}{2} v(t + \Dt/2) \\
v(t + 3\Dt/4) &=& v(t + \Dt/2) + \frac{b\Dt}{2m} f(t+\Dt) \\
v(t + \Dt) &=& \sqrt{a}v(t+ 3\Dt/4) + \sqrt{\frac{1-a}{\beta m}} \mathcal{N}^{-}(t+\Dt)
\eea
Where $\Lambda(t)$ is the method by which the Hamiltonian is
updated. The $b$ factor is more generally set to 1; we will return to
it later. 

This can be condensed somewhat. In the case that Hamiltonian does not
change the mass, the Hamiltonian and position updates commute, and can
be condensed (if $b$ is independent of mass, it will commute anyway).

\bea
v(t + \Dt/2) &=& \sqrt{a}v(t) + \sqrt{\frac{1-a}{\beta m}} \mathcal{N}^{+}(t) + \frac{b\Dt}{2m} f(t) \\
x(t + \Dt) &=& x(t) + b\Dt v(t + \Dt/2) \\
\mathcal(H)(t+\Delta t) = \Lambda(t) \\
v(t + \Dt) &=& \sqrt{a}v(t + \Dt/2) + \sqrt{a}\frac{b\Dt}{2m} f(t+\Dt) + \sqrt{\frac{1-a}{\beta m}} \mathcal{N}^{-}(t+\Dt) 
\eea

It turns out that having two separate random processes is necessary
for irreducibility of the Markov process.  In many cases, however,
this will not be required.  We can then write a leapfrog version:
\bea
v(t + \Dt/2) &=& a v(t - \Dt/2) + (1+a)\frac{b\Dt}{2m} f(t+\Dt) + \sqrt{\frac{1-a^2}{\beta m}}\mathcal{N}(t) \\
x(t + \Dt) &=& x(t) + b\Dt v(t + \Dt/2) \\
\mathcal(H)(t+\Delta t) = \Lambda(t)
\eea
Where the full stochastic step is treated as a stochastic step of
length $2\Dt$, rather than repeating the term twice.

% } % Brace matches ifthenelse test for gmxlite

% \ifthenelse{\equal{\gmxlite}{1}}{}{

\section{Other Monte Carlo routines}

Given the general organization, defining new Monte Carlo routines will
be straightforward.  

\section{Grand canonical Monte Carlo}

Grand canonical Monte Carlo ($\mu$VT ensemble) involves a changing
number of particles.  Given the current code restraints in \gromacs,
this seems simply too difficult to do, without sufficient return.

In theory, this would be possible to do by maintaining a reservoir of
particles that float in the main volume, but do not interact, but this
would be messy.

\section{Gibbs ensemble Monte Carlo} 

A better way to deal with varying the number of particles in the system
is to use Gibbs ensemble Monte Carlo. This can be seen as a way to
``glue'' two grand canonical Monte Carlo systems that are at equal
chemical potentials, or that optionally are offset by some constant
chemical potential.

This involved keeping track of a fixed number of total particles (so
each frame contains the same number of particles), but two containers.
Each container starts at a different phase, and the phases are in
contact with each other via a nonphysical insertion/deletion process
rather than along an interface.  

In order to implement this in GROMACS, it would require 
\begin{itemize}
\item Maintaining two boxes
\item Maintaining a list of membership of each of the boxes
\item introducing Monte Carlo moves between the two boxes
\item Routines for separating the trajectories into the two boxes
\end{itemize}

This is much more doable than changing the number of particles, but
still would require significant reorganization efforts that may
complicate the codes.  I would suggest that this may be something for
5.1 or beyond.

\section{General ensemble methods}

\subsection{Replica exchange\index{replica exchange}}
Replica exchange molecular dynamics (\normindex{REMD})
is a method that can be used to speed up
the sampling of any type of simulation, especially if
conformations are separated by relatively high energy barriers.
It involves simulating multiple replicas of the same system
at different temperatures and randomly exchanging the complete state
of two replicas at regular intervals with the probability:
\beq
P(1 \leftrightarrow 2)=\min\left(1,\exp\left[
\left(\frac{1}{k_B T_1} - \frac{1}{k_B T_2}\right)(U_1 - U_2)
 \right] \right)
\eeq
where $T_1$ and $T_2$ are the reference temperatures and $U_1$ and $U_2$
are the instantaneous potential energies of replicas 1 and 2 respectively.
After exchange the velocities are scaled by $(T_1/T_2)^{\pm0.5}$
and a neighbor search is performed the next step.
This combines the fast sampling and frequent barrier-crossing
of the highest temperature with correct Boltzmann sampling at
all the different temperatures~\cite{Hukushima96a,Sugita99}.
We only attempt exchanges for neighboring temperatures as the probability
decreases very rapidly with the temperature difference.
One should not attempt exchanges for all possible pairs in one step.
If, for instance, replicas 1 and 2 would exchange, the chance of
exchange for replicas 2 and 3 not only depends on the energies of
replicas 2 and 3, but also on the energy of replica 1.
In {\gromacs} this is solved by attempting exchange for all ``odd''
pairs on ``odd'' attempts and for all ``even'' pairs on ``even'' attempts.
If we have four replicas: 0, 1, 2 and 3, ordered in temperature
and we attempt exchange every 1000 steps, pairs 0-1 and 2-3
will be tried at steps 1000, 3000 etc. and pair 1-2 at steps 2000, 4000 etc.

How should one choose the temperatures?
The energy difference can be written as:
\beq
U_1 - U_2 =  N_{df} \frac{c}{2} k_B (T_1 - T_2)
\eeq
where $N_{df}$ is the total number of degrees of freedom of one replica
and $c$ is 1 for harmonic potentials and around 2 for protein/water systems.
If $T_2 = (1+\epsilon) T_1$ the probability becomes:
\beq
P(1 \leftrightarrow 2)
  = \exp\left( -\frac{\epsilon^2 c\,N_{df}}{2 (1+\epsilon)} \right)
\approx \exp\left(-\epsilon^2 \frac{c}{2} N_{df} \right)
\eeq
Thus for a probability of $e^{-2}\approx 0.135$
one obtains $\epsilon \approx 2/\sqrt{c\,N_{df}}$.
With all bonds constrained one has $N_{df} \approx 2\, N_{atoms}$
and thus for $c$ = 2 one should choose $\epsilon$ as $1/\sqrt{N_{atoms}}$.
However there is one problem when using pressure coupling. The density at
higher temperatures will decrease, leading to higher energy~\cite{Seibert2005a},
which should be taken into account. The {\gromacs} website features a
so-called ``REMD calculator,'' that lets you type in the temperature range and
the number of atoms, and based on that proposes a set of temperatures.

An extension to the REMD for the isobaric-isothermal ensemble was
proposed by Okabe {\em et al.}~\cite{Okabe2001a}. In this work the
exchange probability is modified to:
\beq
P(1 \leftrightarrow 2)=\min\left(1,\exp\left[
\left(\frac{1}{k_B T_1} - \frac{1}{k_B T_2}\right)(U_1 - U_2) +
\left(\frac{P_1}{k_B T_1} - \frac{P_2}{k_B T_2}\right)\left(V_1-V_2\right)
 \right] \right)
\eeq
where $P_1$ and $P_2$ are the respective reference pressures and $V_1$ and
$V_2$ are the respective instantaneous volumes in the simulations.
In most cases the differences in volume are so small that the second
term is negligible. It only plays a role when the difference between
$P_1$ and $P_2$ is large or in phase transitions.

Hamiltonian replica exchange is also supported in {\gromacs}.  In
Hamiltonian replica exchange, each replica has a different
Hamiltonian, defined by the free energy pathway specified for the simulation.  The
exchange probability to maintain the correct ensemble probabilities is:
\beq P(1 \leftrightarrow 2)=\min\left(1,\exp\left[
    \left(\frac{1}{k_B T} - \frac{1}{k_B T}\right)((U_1(x_2) - U_1(x_1)) + (U_2(x_1) - U_2(x_2)))
\right]
\right)
\eeq
The separate Hamiltonians are defined by the free energy functionality
of Gromacs, with swaps made between the different values of
$\lambda$ defined in the mdp file.

Hamiltonian and temperature replica exchange can also be performed
simultaneously, using the acceptance criteria:
\beq
P(1 \leftrightarrow 2)=\min\left(1,\exp\left[
\left(\frac{1}{k_B T} - \right)(\frac{U_1(x_2) - U_1(x_1)}{k_B T_1} + \frac{U_2(x_1) - U_2(x_2)}{k_B T_2})
 \right] \right)
\eeq

Gibbs sampling replica exchange has also been implemented in
{\gromacs}~\cite{Chodera2011}.  In Gibbs sampling replica exchange, all
possible pairs are tested for exchange, allowing swaps between
replicas that are not neighbors.

Gibbs sampling replica exchange requires no additional potential
energy calculations.  However there is an additional communication
cost in Gibbs sampling replica exchange, as for some permutations,
more than one round of swaps must take place.  In some cases, this
extra communication cost might affect the efficiency.

All replica exchange variants are options of the {\tt mdrun}
program. It will only work when MPI is installed, due to the inherent
parallelism in the algorithm. For efficiency each replica can run on a
separate node.  See the manual page of {\tt mdrun} on how to use these
multinode features.

\subsection{Expanded ensemble}

To be filled in later.  This is expected to be essentially the same as in 4.6.

\subsection{$\lambda$ dynamics}

I think there are advantages for having $\lambda$ dynamics, as well as
having discrete states.  The biggest reason is that when making moves
in joint coordinate/$\lambda$ space, simultaneous moves in both
dimensions are usually more efficient than moves alternating
dimensions.  This will require more thinking and work.

\bibliographystyle{proteins}
\bibliography{monster}
\end{document}
% LocalWords:  GROningen MAchine BIOSON Groningen GROMACS Berendsen der Spoel
% LocalWords:  Drunen Comp Phys Comm ROck NS FFT pbc EM ifthenelse gmxlite ff
% LocalWords:  octahedra triclinic Ewald PME PPPM Gromacs trjconv xy solvated
% LocalWords:  boxtypes boxshapes editconf Lennard mdpopt COM XTC kT defunits
% LocalWords:  Boltzmann's Mueller nb int mdrun chargegroup simplerc prefactor
% LocalWords:  pme waterloops CH NH CO df com virial integrator Verlet vverlet
% LocalWords:  integrators ref timepoint timestep timesteps mdp md vv avek NVE
% LocalWords:  NVT off's leapfrogv lll LR rmfast SPC fs Nos physicality ps GMX
% LocalWords:  Tcoupling nonergodic thermostatting NOSEHOOVER algorithmes ij yx
% LocalWords:  Parrinello Rahman rescales atm anisotropically ccc xz zx yy yz
% LocalWords:  zy zz se barostat compressibilities MTTK NPT Martyna al isobaric
% LocalWords:  Tuckerman vir PV fkT iLt iL Liouville NHC Eq baro mu trj mol bc
% LocalWords:  freezegroup Shannon's polarizability Overhauser barostats iLn KE
% LocalWords:  negligibly thermostatted Tobias  rhombic maxwell et xtc TC rlist
% LocalWords:  waals LINCS holonomic plincs lincs unc ang SA Langevin SD amu BD
% LocalWords:  bfgs Broyden Goldfarb Shanno mkT kJ DFLEXIBLE Nocedal diag nmeig
% LocalWords:  diagonalization anaeig nmens covanal ddg feia BT dp dq pV dV dA
% LocalWords:  NpT eq stepsize REMD constrainted website Okabe MPI covar edi dd
% LocalWords:  progman NMR ddcells innerloops ddtric tric dds rdd conf rcon est
% LocalWords:  mb PP MPMD ddorder pp cartesian grompp npme parallelizable edr
% LocalWords:  macromolecule nstlist vacuo parallelization dof indices MBAR AVX
% LocalWords:  TOL numerics parallelized eigenvectors dG parallelepipeds VdW np
% LocalWords:  Coul multi solvation HCT OBC solv cav vdw Schaefer symplectic dt
% LocalWords:  pymbar multinode subensemble Monte solute subst groupconcept GPU
% LocalWords:  dodecahedron octahedron dodecahedra equilibration usinggroups nm
% LocalWords:  topologies rlistlong CUDA GPUs rcoulomb SIMD BlueGene FPUs erfc
% LocalWords:  cutoffschemesupport unbuffered bondeds AdResS OpenMP ewald rtol
% LocalWords:  verletdrift peptide RMS rescaling ergodicity ergodic discretized
% LocalWords:  isothermal compressibility isotropically anisotropic iteratively
% LocalWords:  incompressible integrations translational biomolecules NMA PCA
% LocalWords:  Bennett's equilibrated Hamiltonians covariance equilibrate ji PR
% LocalWords:  inhomogeneous conformational textwidth eqn eqns sec incrementors
% LocalWords:  OVRVO Sivak Bussi equipartition Multistep inL realspace FSR FLR
% LocalWords:  FVSR commutivity neighborlist resample multistep VT nonphysical
% LocalWords:  ize Leimkuhler Stat stoc dW Nonisotropic SCC irreducibility av
