%
% 
%       This source code is part of
% 
%        G   R   O   M   A   C   S
% 
% GROningen MAchine for Chemical Simulations
% 
%               VERSION 2.0
% 
% Copyright (c) 1991-1999
% BIOSON Research Institute, Dept. of Biophysical Chemistry
% University of Groningen, The Netherlands
% 
% Please refer to:
% GROMACS: A message-passing parallel molecular dynamics implementation
% H.J.C. Berendsen, D. van der Spoel and R. van Drunen
% Comp. Phys. Comm. 91, 43-56 (1995)
% 
% Also check out our WWW page:
% http://md.chem.rug.nl/~gmx
% or e-mail to:
% gromacs@chem.rug.nl
% 
% And Hey:
% Giving Russians Opium May Alter Current Situation
%

\chapter{Run parameters and Programs}
\label{ch:programs}

\section{On-line and HTML manuals}
\index{online manual}
\index{html manual}
All the information in this chapter can also be found in HTML
format in your GROMACS data directory. The path depends on
where your files are installed, but the default location is \\
\centerline{\tt /usr/local/gromacs/share/html/online.html}
Or, if you installed from Linux packages it can be found as\\
\centerline{\tt /usr/local/share/gromacs/html/online.html}
You can also use the online from our web site,\\
\centerline{\href{http://manual.gromacs.org/current/}{http://manual.gromacs.org/current/}}

In addition, we install standard UNIX manuals for all the programs. If
you have sourced the {\tt GMXRC} script in the GROMACS binary directory for
your host they should already be present in your {\tt \$MANPATH}, and you
should be able to type {\eg} {\tt man grompp}.

The program manual pages can also be found in
\appref{progman} in this manual.

\section{File types\swapindexquiet{file}{type}}
\label{sec:fileformats}
\tabref{form} lists the file types used by {\gromacs} along with
a short description, and you can find a more detail description for
each file in your HTML reference, or in our online version.

{\gromacs} files written in \normindex{XDR} format can be read on any
architecture with {\gromacs} version 1.6 or later if the configuration
script found the XDR libraries on your system. They should always be
present on UNIX since they are necessary for NFS support.

% This file is part of the GROMACS molecular simulation package.
%
% Copyright (c) 2013, by the GROMACS development team, led by
% David van der Spoel, Berk Hess, Erik Lindahl, and including many
% others, as listed in the AUTHORS file in the top-level source
% directory and at http://www.gromacs.org.
%
% GROMACS is free software; you can redistribute it and/or
% modify it under the terms of the GNU Lesser General Public License
% as published by the Free Software Foundation; either version 2.1
% of the License, or (at your option) any later version.
%
% GROMACS is distributed in the hope that it will be useful,
% but WITHOUT ANY WARRANTY; without even the implied warranty of
% MERCHANTABILITY or FITNESS FOR A PARTICULAR PURPOSE. See the GNU
% Lesser General Public License for more details.
%
% You should have received a copy of the GNU Lesser General Public
% License along with GROMACS; if not, see
% http://www.gnu.org/licenses, or write to the Free Software Foundation,
% Inc., 51 Franklin Street, Fifth Floor, Boston, MA  02110-1301  USA.
%
% If you want to redistribute modifications to GROMACS, please
% consider that scientific software is very special. Version
% control is crucial - bugs must be traceable. We will be happy to
% consider code for inclusion in the official distribution, but
% derived work must not be called official GROMACS. Details are found
% in the README & COPYING files - if they are missing, get the
% official version at http://www.gromacs.org.
%
% To help us fund GROMACS development, we humbly ask that you cite
% the research papers on the package. Check out http://www.gromacs.org

\begin{table}
\begin{tabularx}{\linewidth}{|r@{\tt.}lccX|}
\dline
\mc{2}{|c}{Default} &      & Default &  \\[-0.1ex]
\mc{1}{|c}{Name} & \mc{1}{c}{Ext.} & Type &  Option & Description \\[-0.1ex]
\hline
\tt   atomtp & \tt atp & Asc & \tt    & Atomtype file used by {\tt pdb2gmx} \\[-0.1ex]
\tt    eiwit & \tt brk & Asc & \tt -f & Brookhaven data bank file \\[-0.1ex]
\tt    state & \tt cpt & xdr & \tt    & Checkpoint file \\[-0.1ex]
\tt   nnnice & \tt dat & Asc & \tt    & Generic data file \\[-0.1ex]
\tt     user & \tt dlg & Asc & \tt    & Dialog Box data for {\tt ngmx} \\[-0.1ex]
\tt      sam & \tt edi & Asc & \tt    & ED sampling input \\[-0.1ex]
\tt      sam & \tt edo & Asc & \tt    & ED sampling output \\[-0.1ex]
\tt     ener & \tt edr &     & \tt    & Generic energy: \tt edr ene \\[-0.1ex]
\tt     ener & \tt edr & xdr & \tt    & Energy file in portable xdr format \\[-0.1ex]
\tt     ener & \tt ene & Bin & \tt    & Energy file \\[-0.1ex]
\tt    eiwit & \tt ent & Asc & \tt -f & Entry in the protein date bank \\[-0.1ex]
\tt     plot & \tt eps & Asc & \tt    & Encapsulated PostScript (tm) file \\[-0.1ex]
\tt     conf & \tt esp & Asc & \tt -c & Coordinate file in ESPResSo format \\[-0.1ex]
\tt    gtraj & \tt g87 & Asc & \tt    & Gromos-87 ASCII trajectory format \\[-0.1ex]
\tt     conf & \tt g96 & Asc & \tt -c & Coordinate file in Gromos-96 format \\[-0.1ex]
\tt     conf & \tt gro & Asc & \tt -c & Coordinate file in Gromos-87 format \\[-0.1ex]
\tt     conf & \tt gro &     & \tt -c & Structure: \tt gro g96 pdb esp tpr tpb tpa \\[-0.1ex]
\tt      out & \tt gro &     & \tt -o & Structure: \tt gro g96 pdb esp \\[-0.1ex]
\tt    polar & \tt hdb & Asc & \tt    & Hydrogen data base \\[-0.1ex]
\tt   topinc & \tt itp & Asc & \tt    & Include file for topology \\[-0.1ex]
\tt      run & \tt log & Asc & \tt -l & Log file \\[-0.1ex]
\tt       ps & \tt m2p & Asc & \tt    & Input file for mat2ps \\[-0.1ex]
\tt       ss & \tt map & Asc & \tt    & File that maps matrix data to colors \\[-0.1ex]
\tt       ss & \tt mat & Asc & \tt    & Matrix Data file \\[-0.1ex]
\tt   grompp & \tt mdp & Asc & \tt -f & {\tt grompp} input file with MD parameters \\[-0.1ex]
\tt  hessian & \tt mtx & Bin & \tt -m & Hessian matrix \\[-0.1ex]
\tt    index & \tt ndx & Asc & \tt -n & Index file \\[-0.1ex]
\tt    hello & \tt out & Asc & \tt -o & Generic output file \\[-0.1ex]
\tt    eiwit & \tt pdb & Asc & \tt -f & Protein data bank file \\[-0.1ex]
\tt  residue & \tt rtp & Asc & \tt    & Residue Type file used by {\tt pdb2gmx} \\[-0.1ex]
\tt      doc & \tt tex & Asc & \tt -o & LaTeX file \\[-0.1ex]
\tt    topol & \tt top & Asc & \tt -p & Topology file \\[-0.1ex]
\tt    topol & \tt tpb & Bin & \tt -s & Binary run input file \\[-0.1ex]
\tt    topol & \tt tpr &     & \tt -s & Generic run input: \tt tpr tpb tpa \\[-0.1ex]
\tt    topol & \tt tpr &     & \tt -s & Structure+mass(db): \tt tpr tpb tpa gro g96 pdb \\[-0.1ex]
\tt    topol & \tt tpr & xdr & \tt -s & Portable xdr run input file \\[-0.1ex]
\tt     traj & \tt trj & Bin & \tt    & Trajectory file (architecture specific) \\[-0.1ex]
\tt     traj & \tt trr &     & \tt    & Full precision trajectory: \tt trr trj cpt \\[-0.1ex]
\tt     traj & \tt trr & xdr & \tt    & Trajectory in portable xdr format \\[-0.1ex]
\tt     root & \tt xpm & Asc & \tt    & X PixMap compatible matrix file \\[-0.1ex]
\tt     traj & \tt xtc &     & \tt -f & Trajec., input: \tt xtc trr trj cpt gro g96 pdb \\[-0.1ex]
\tt     traj & \tt xtc &     & \tt -f & Trajectory, output: \tt xtc trr trj gro g96 pdb \\[-0.1ex]
\tt     traj & \tt xtc & xdr & \tt    & Compressed trajectory (portable xdr format) \\[-0.1ex]
\tt    graph & \tt xvg & Asc & \tt -o & xvgr/xmgr file \\[-0.1ex]
\dline
\end{tabularx}
\caption{The {\gromacs} file types.}
\label{tab:form}
\end{table}

% LocalWords:  lccX atomtp atp Asc Atomtype pdb gmx eiwit brk Brookhaven cpt tm
% LocalWords:  xdr nnnice dat dlg ngmx sam edi edo ener edr ene ent eps conf ss
% LocalWords:  PostScript ESPResSo gtraj Gromos gro tpr tpb tpa hdb topinc itp
% LocalWords:  grompp mdp mtx ndx rtp tex LaTeX topol traj trj trr xpm PixMap
% LocalWords:  xtc Trajec xvg xvgr xmgr


\section{Run Parameters\swapindexquiet{run}{parameter}}
\input{mdp_opt}

\section{Programs by topic\index{programs by topic}}
\input{proglist}

% LocalWords:  online html GMXRC MANPATH grompp progman xdr NFS
