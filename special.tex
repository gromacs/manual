%
% 
%       This source code is part of
% 
%        G   R   O   M   A   C   S
% 
% GROningen MAchine for Chemical Simulations
% 
%               VERSION 2.0
% 
% Copyright (c) 1991-1999
% BIOSON Research Institute, Dept. of Biophysical Chemistry
% University of Groningen, The Netherlands
% 
% Please refer to:
% GROMACS: A message-passing parallel molecular dynamics implementation
% H.J.C. Berendsen, D. van der Spoel and R. van Drunen
% Comp. Phys. Comm. 91, 43-56 (1995)
% 
% Also check out our WWW page:
% http://md.chem.rug.nl/~gmx
% or e-mail to:
% gromacs@chem.rug.nl
% 
% And Hey:
% Giving Russians Opium May Alter Current Situation
%

\chapter{Special Topics}
\label{ch:special}

\section{Potential of mean force}

A potential of mean force (PMF) is a potential that is obtained
by integrating the mean force from an ensemble of configurations.
In {\gromacs}, there are several different methods to calculate the mean force.
Each method has its limitations, which are listed below.
\begin{itemize}
\item{\bf pull code:} between the centers of mass of molecules or groups of molecules.
\item{\bf free-energy code with harmonic bonds or constraints:} between single atoms. 
\item{\bf free-energy code with position restraints:} changing the conformation of a relatively immobile group of atoms.
\item{\bf pull code in limited cases:} between groups of atoms that are
part of a larger molecule for which the bonds are constrained with
SHAKE or LINCS. If the pull group if relatively large,
the pull code can be used.
\end{itemize}
The pull and free-energy code a described in more detail
in the following two sections.

\subsubsection{Entropic effects}
When a distance between two atoms or the centers of mass of two groups
is constrained or restrained, there will be a purely entropic contribution
to the PMF due to the rotation of the two groups~\cite{RMNeumann1980a}.
For a system of two non-interacting masses the potential of mean force is:
\beq
V_{pmf}(r) = -(n_c - 1) k_B T \log(r)
\eeq
where $n_c$ is the number of dimensions in which the constraint works
(i.e. $n_c=3$ for a normal constraint and $n_c=1$ when only
the $z$-direction is constrained).
Whether one needs to correct for this contribution depends on what
the PMF should represent. When one wants to pull a substrate
into a protein, this entropic term indeed contributes to the work to
get the substrate into the protein. But when calculating a PMF
between two solutes in a solvent, for the purpose of simulating
without solvent, the entropic contribution should be removed.
{\bf Note} that this term can be significant; when at 300K the distance is halved,
the contribution is 3.5 kJ~mol$^{-1}$.

\section{Non-equilibrium pulling}
When the distance between two groups is changed continuously,
work is applied to the system, which means that the system is no longer
in equilibrium. Although in the limit of very slow pulling
the system is again in equilibrium, for many systems this limit
is not reachable within reasonable computational time.
However, one can use the Jarzynski relation~\cite{Jarzynski1997a}
to obtain the equilibrium free-energy difference $\Delta G$
between two distances from many non-equilibrium simulations:
\begin{equation}
   \Delta G_{AB} = -k_BT \log \left\langle e^{-\beta W_{AB}} \right\rangle_A
   \label{eq:Jarz}
\end{equation}
where $W_{AB}$ is the work performed to force the system along one path
from state A to B, the angular bracket denotes averaging over
a canonical ensemble of the initial state A and $\beta=1/k_B T$.


\section{The pull code}
\index{center-of-mass pulling}
\label{sec:pull}
The pull code applies forces or constraints between the centers
of mass of one or more pairs of groups of atoms.
There is one reference group and one or more other pull groups.
Instead of a reference group, one can also use absolute reference
point in space.
The most common situation consists of a reference group
and one pull group. In this case, the two groups are treated
equivalently.
The distance between a pair of groups can be determined
in 1, 2 or 3 dimensions, or can be along a user-defined vector.
The reference distance can be constant or can change linearly with time.
Normally all atoms are weighted by their mass, but an additional
weighting factor can also be used.
\begin{figure}
\centerline{\includegraphics[width=6cm,angle=270]{plots/pull}}
\caption{Schematic picture of pulling a lipid out of a lipid bilayer
with umbrella pulling. $V_{rup}$ is the velocity at which the spring is
retracted, $Z_{link}$ is the atom to which the spring is attached and
$Z_{spring}$ is the location of the spring.}
\label{fi:pull} 
\end{figure}

Three different types of calculation are supported,
and in all cases the reference distance can be constant
or linearly changing with time.
\begin{enumerate}
\item{\textbf{\swapindex{Umbrella}{pulling}}}
A harmonic potential is applied between
the centers of mass of two groups.
Thus, the force is proportional to the displacement.
\item{\textbf{\swapindex{Constraint}{pulling}}}
The distance between the centers of mass of two groups is constrained.
The constraint force can be written to a file.
This method uses the SHAKE algorithm but only needs 1 iteration to be
exact if only two groups are constrained. 
\item{\textbf{Constant force pulling}}
A constant force is applied between the centers of mass of two groups.
Thus, the potential is linear.
In this case there is no reference distance of pull rate.
\end{enumerate}

\subsubsection{Definition of the center of mass}

In {\gromacs}, there are three ways to define the center of mass of a group.
The standard way is a ``plain'' center of mass, possibly with additional
weighting factors. With periodic boundary conditions it is no longer
possible to uniquely define the center of mass of a group of atoms.
Therefore, a reference atom is used. For determining the center of mass,
for all other atoms in the group, the closest periodic image to the reference
atom is used. This uniquely defines the center of mass.
By default, the middle (determined by the order in the topology) atom
is used as a reference atom, but the user can also select any other atom
if it would be closer to center of the group.

For a layered system, for instance a lipid bilayer, it may be of interest
to calculate the PMF of a lipid as function of its distance
from the whole bilayer. The whole bilayer can be taken as reference
group in that case, but it might also be of interest to define the
reaction coordinate for the PMF more locally. The {\tt .mdp} option
{\tt pull_geometry = cylinder} does not
use all the atoms of the reference group, but instead dynamically only those
within a cylinder with radius {\tt r_1} around the pull vector going
through the pull group. This only
works for distances defined in one dimension, and the cylinder is
oriented with its long axis along this one dimension. A second cylinder
can be defined with {\tt r_0}, with a linear switch function that weighs
the contribution of atoms between {\tt r_0} and {\tt r_1} with
distance. This smooths the effects of atoms moving in and out of the
cylinder (which causes jumps in the pull forces).

\begin{figure}
\centerline{\includegraphics[width=6cm]{plots/pullref}}
\caption{Comparison of a plain center of mass reference group versus a cylinder
reference group applied to interface systems. C is the reference group.
The circles represent the center of mass of two groups plus the reference group,
$d_c$ is the reference distance.}
\label{fi:pullref} 
\end{figure}   

For a group of molecules in a periodic system, a plain reference group
might not be well-defined. An example is a water slab that is connected
periodically in $x$ and $y$, but has two liquid-vapor interfaces along $z$.
In such a setup, water molecules can evaporate from the liquid and they
will move through the vapor, through the periodic boundary, to the other
interface. Such a system is inherently periodic and there is no proper way
of defining a ``plain'' center of mass along $z$. A proper solution is to using
a cosine shaped weighting profile for all atoms in the reference group.
The profile is a cosine with a single period in the unit cell. Its phase
is optimized to give the maximum sum of weights, including mass weighting.
This provides a unique and continuous reference position that is nearly
identical to the plain center of mass position in case all atoms are all
within a half of the unit-cell length. See ref \cite{Engin2010a} for details.

When relative weights $w_i$ are used during the calculations, either
by supplying weights in the input or due to cylinder geometry
or due to cosine weighting,
the weights need to be scaled to conserve momentum:
\beq
w'_i = w_i
\left. \sum_{j=1}^N w_j \, m_j \right/ \sum_{j=1}^N w_j^2 \, m_j
\eeq
where $m_j$ is the mass of atom $j$ of the group.
The mass of the group, required for calculating the constraint force, is:
\beq
M = \sum_{i=1}^N w'_i \, m_i
\eeq
The definition of the weighted center of mass is:
\beq
\ve{r}_{com} = \left. \sum_{i=1}^N w'_i \, m_i \, \ve{r}_i \right/ M
\eeq
From the centers of mass the AFM, constraint, or umbrella force $\ve{F}_{\!com}$
on each group can be calculated.
The force on the center of mass of a group is redistributed to the atoms
as follows:
\beq
\ve{F}_{\!i} = \frac{w'_i \, m_i}{M} \, \ve{F}_{\!com}
\eeq

\subsubsection{Limitations}
There is one important limitation:
strictly speaking, constraint forces can only be calculated between
groups that are not connected by constraints to the rest of the system.
If a group contains part of a molecule of which the bond lengths
are constrained, the pull constraint and LINCS or SHAKE bond constraint
algorithms should be iterated simultaneously. This is not done in {\gromacs}.
This means that for simulations with {\tt constraints = all-bonds}
in the {\tt .mdp} file pulling is, strictly speaking,
limited to whole molecules or groups of molecules.
In some cases this limitation can be avoided by using the free energy code,
see \secref{fepmf}.
In practice, the errors caused by not iterating the two constraint
algorithms can be negligible when the pull group consists of a large
amount of atoms and/or the pull force is small.
In such cases, the constraint correction displacement of the pull group
is small compared to the bond lengths.


\section{Calculating a PMF using the free-energy code}
\label{sec:fepmf}
\index{potentials of mean force}
\index{free energy calculations}
The free-energy coupling-parameter approach (see~\secref{fecalc})
provides several ways to calculate potentials of mean force.
A potential of mean force between two atoms can be calculated
by connecting them with a harmonic potential or a constraint.
For this purpose there are special potentials that avoid the generation of
extra exclusions, see~\secref{excl}.
When the position of the minimum or the constraint length is 1 nm more
in state B than in state A, the restraint or constraint force is given
by $\partial H/\partial \lambda$.
The distance between the atoms can be changed as a function of $\lambda$
and time by setting {\tt delta-lambda} in the {\tt .mdp} file.
The results should be identical (although not numerically
due to the different implementations) to the results of the pull code
with umbrella sampling and constraint pulling.
Unlike the pull code, the free energy code can also handle atoms that
are connected by constraints.

Potentials of mean force can also be calculated using position restraints.
With position restraints, atoms can be linked to a position in space
with a harmonic potential (see \secref{posre}).
These positions can be made a function of the coupling parameter $\lambda$.
The positions for the A and the B states are supplied to {\tt grompp} with
the {\tt -r} and {\tt -rb} options, respectively.
One could use this approach to do \normindex{targeted MD};
note that we do not encourage the use of targeted MD for proteins.
A protein can be forced from one conformation to another by using
these conformations as position restraint coordinates for state A and B.
One can then slowly change $\lambda$ from 0 to 1.
The main drawback of this approach is that the conformational freedom
of the protein is severely limited by the position restraints,
independent of the change from state A to B.
Also, the protein is forced from state A to B in an almost straight line,
whereas the real pathway might be very different.
An example of a more fruitful application is a solid system or a liquid
confined between walls where one wants to measure the force required
to change the separation between the boundaries or walls.
Because the boundaries (or walls) already need to be fixed,
the position restraints do not limit the system in its sampling.

%%%%%%%%%%%%%%%%%%%%%%%%%%%%%%%%%%%%%%%%%%%%%%%%%%%%%%%%%%%%%%%%%%%%%%%%%%%%%%%
%%%%%%%%%%%%%%%%%%%%%%%%%%%%%%%%%%%%%%%%%%%%%%%%%%%%%%%%%%%%%%%%%%%%%%%%%%%%%%%
%%%%%%%%%%%%%%%%%%%%%%%%%%%%%%%%%%%%%%%%%%%%%%%%%%%%%%%%%%%%%%%%%%%%%%%%%%%%%%%
\newcommand{\amine}{\sf -NH$_2$}
\newcommand{\amines}{\sf -NH-}
\newcommand{\aminep}{\sf -NH$_3^+$}
\section{Removing fastest \swapindex{degrees of}{freedom}}
\label{sec:rmfast}
The maximum time step in MD simulations is limited by the smallest
oscillation period that can be found in the simulated
system. Bond-stretching vibrations are in their quantum-mechanical
ground state and are therefore better represented by a constraint 
instead of a harmonic potential.

For the remaining degrees of freedom, the shortest oscillation period
(as measured from a simulation) is 13~fs for bond-angle vibrations
involving hydrogen atoms. Taking as a guideline that with a Verlet
(leap-frog) integration scheme a minimum of 5 numerical integration
steps should be performed per period of a harmonic oscillation in
order to integrate it with reasonable accuracy, the maximum time step
will be about 3~fs. Disregarding these very fast oscillations of
period 13~fs, the next shortest periods are around 20~fs, which will
allow a maximum time step of about 4~fs.

Removing the bond-angle degrees of freedom from hydrogen atoms can
best be done by defining them as \normindex{virtual interaction-sites}
instead of normal atoms. Whereas a normal atom is connected to the molecule
with bonds, angles and dihedrals, a virtual site's position is calculated
from the position of three nearby heavy atoms in a predefined manner
(see also \secref{virtual_sites}). For the hydrogens in water and in
hydroxyl, sulfhydryl, or amine groups, no degrees of freedom can be
removed, because rotational freedom should be preserved. The only
other option available to slow down these motions is to increase the
mass of the hydrogen atoms at the expense of the mass of the connected
heavy atom. This will increase the moment of inertia of the water
molecules and the hydroxyl, sulfhydryl, or amine groups, without
affecting the equilibrium properties of the system and without
affecting the dynamical properties too much. These constructions will
shortly be described in \secref{vsitehydro} and have previously
been described in full detail~\cite{feenstra99}.

Using both virtual sites and \swapindex{modified}{mass}es, the next
bottleneck is likely to be formed by the improper dihedrals (which are
used to preserve planarity or chirality of molecular groups) and the
peptide dihedrals. The peptide dihedral cannot be changed without
affecting the physical behavior of the protein. The improper dihedrals
that preserve planarity mostly deal with aromatic residues. Bonds,
angles, and dihedrals in these residues can also be replaced with
somewhat elaborate virtual site constructions.

All modifications described in this section can be performed using the
{\gromacs} topology building tool {\tt \normindex{pdb2gmx}}. Separate
options exist to increase hydrogen masses, virtualize all hydrogen atoms,
or also virtualize all aromatic residues. {\bf Note} that when all hydrogen
atoms are virtualized, those inside the aromatic residues will be
virtualized as well, {\ie} hydrogens in the aromatic residues are treated
differently depending on the treatment of the aromatic residues.

Parameters for the virtual site constructions for the hydrogen atoms are
inferred from the force field parameters ({\em vis}. bond lengths and
angles) directly by {\tt \normindex{grompp}} while processing the
topology file.  The constructions for the aromatic residues are based
on the bond lengths and angles for the geometry as described in the
force fields, but these parameters are hard-coded into {\tt
\normindex{pdb2gmx}} due to the complex nature of the construction
needed for a whole aromatic group.

\subsection{Hydrogen bond-angle vibrations}
\label{sec:vsitehydro}
\subsubsection{Construction of virtual sites} %%%%%%%%%%%%%%%%%%%%%%%%%
\begin{figure}
\centerline{\includegraphics[width=11cm]{plots/dumtypes}}
\caption[Virtual site constructions for hydrogen atoms.]{The different
types of virtual site constructions used for hydrogen atoms. The atoms
used in the construction of the virtual site(s) are depicted as black
circles, virtual sites as gray ones. Hydrogens are smaller than heavy
atoms. {\sf A}: fixed bond angle, note that here the hydrogen is not a
virtual site; {\sf B}: in the plane of three atoms, with fixed distance;
{\sf C}: in the plane of three atoms, with fixed angle and distance;
{\sf D}: construction for amine groups ({\amine} or {\aminep}), see
text for details.}
\label{fig:vsitehydro}
\end{figure}

The goal of defining hydrogen atoms as virtual sites is to remove all
high-frequency degrees of freedom from them. In some cases, not all
degrees of freedom of a hydrogen atom should be removed, {\eg} in the
case of hydroxyl or amine groups the rotational freedom of the
hydrogen atom(s) should be preserved. Care should be taken that no
unwanted correlations are introduced by the construction of virtual
sites, {\eg} bond-angle vibration between the constructing atoms could
translate into hydrogen bond-length vibration. Additionally, since
virtual sites are by definition massless, in order to preserve total
system mass, the mass of each hydrogen atom that is treated as virtual
site should be added to the bonded heavy atom.

Taking into account these considerations, the hydrogen atoms in a
protein naturally fall into several categories, each requiring a
different approach (see also \figref{vsitehydro}).

\begin{itemize}

\item{\em hydroxyl ({\sf -OH}) or sulfhydryl ({\sf -SH})
hydrogen:\/} The only internal degree of freedom in a hydroxyl group
that can be constrained is the bending of the {\sf C-O-H} angle. This
angle is fixed by defining an additional bond of appropriate length,
see \figref{vsitehydro}A. Doing so removes the high-frequency angle bending,
but leaves the dihedral rotational freedom. The same goes for a
sulfhydryl group. {\bf Note} that in these cases the hydrogen is not treated
as a virtual site.

\item{\em single amine or amide ({\amines}) and aromatic hydrogens
({\sf -CH-}):\/} The position of these hydrogens cannot be constructed
from a linear combination of bond vectors, because of the flexibility
of the angle between the heavy atoms. Instead, the hydrogen atom is
positioned at a fixed distance from the bonded heavy atom on a line
going through the bonded heavy atom and a point on the line through
both second bonded atoms, see \figref{vsitehydro}B.

\item{\em planar amine ({\amine}) hydrogens:\/} The method used for
the single amide hydrogen is not well suited for planar amine groups,
because no suitable two heavy atoms can be found to define the
direction of the hydrogen atoms. Instead, the hydrogen is constructed
at a fixed distance from the nitrogen atom, with a fixed angle to the
carbon atom, in the plane defined by one of the other heavy atoms, see
\figref{vsitehydro}C.

\item{\em amine group (umbrella {\amine} or {\aminep}) hydrogens:\/}
Amine hydrogens with rotational freedom cannot be constructed as virtual
sites from the heavy atoms they are connected to, since this would
result in loss of the rotational freedom of the amine group. To
preserve the rotational freedom while removing the hydrogen bond-angle
degrees of freedom, two ``dummy masses'' are constructed with the same
total mass, moment of inertia (for rotation around the {\sf C-N} bond)
and center of mass as the amine group. These dummy masses have no
interaction with any other atom, except for the fact that they are
connected to the carbon and to each other, resulting in a rigid
triangle. From these three particles, the positions of the nitrogen and
hydrogen atoms are constructed as linear combinations of the two
carbon-mass vectors and their outer product, resulting in an amine
group with rotational freedom intact, but without other internal
degrees of freedom. See \figref{vsitehydro}D.

\end{itemize}

\begin{figure}
\centerline{\includegraphics[width=15cm]{plots/dumaro}}
\caption[Virtual site constructions for aromatic residues.]{The
different types of virtual site constructions used for aromatic
residues. The atoms used in the construction of the virtual site(s) are
depicted as black circles, virtual sites as gray ones. Hydrogens are
smaller than heavy atoms. {\sf A}: phenylalanine; {\sf B}: tyrosine
(note that the hydroxyl hydrogen is {\em not} a virtual site); {\sf C}:
tryptophan; {\sf D}: histidine.}
\label{fig:vistearo}
\end{figure}

\subsection{Out-of-plane vibrations in aromatic groups}
\label{sec:vsitearo}
The planar arrangements in the side chains of the aromatic residues
lends itself perfectly to a virtual-site construction, giving a
perfectly planar group without the inherently unstable constraints
that are necessary to keep normal atoms in a plane. The basic approach
is to define three atoms or dummy masses with constraints between them
to fix the geometry and create the rest of the atoms as simple virtual
sites type (see \secref{virtual_sites}) from these three. Each of
the aromatic residues require a different approach:

\begin{itemize}

\item{\em Phenylalanine:\/} {\sf C}$_\gamma$, {\sf C}$_{{\epsilon}1}$,
and {\sf C}$_{{\epsilon}2}$ are kept as normal atoms, but with each a
mass of one third the total mass of the phenyl group. See
\figref{vsitehydro}A.

\item{\em Tyrosine:\/} The ring is treated identically to the
phenylalanine ring. Additionally, constraints are defined between {\sf
C}$_{{\epsilon}1}$, {\sf C}$_{{\epsilon}2}$, and {\sf O}$_{\eta}$.
The original improper dihedral angles will keep both triangles (one
for the ring and one with {\sf O}$_{\eta}$) in a plane, but due to the
larger moments of inertia this construction will be much more
stable. The bond-angle in the hydroxyl group will be constrained by a
constraint between {\sf C}$_\gamma$ and {\sf H}$_{\eta}$. {\bf Note} that
the hydrogen is not treated as a virtual site. See
\figref{vsitehydro}B.

\item{\em Tryptophan:\/} {\sf C}$_\beta$ is kept as a normal atom
and two dummy masses are created at the center of mass of each of the
rings, each with a mass equal to the total mass of the respective ring
({\sf C}$_{{\delta}2}$ and {\sf C}$_{{\epsilon}2}$ are each
counted half for each ring). This keeps the overall center of mass and
the moment of inertia almost (but not quite) equal to what it was. See
\figref{vsitehydro}C.

\item{\em Histidine:\/} {\sf C}$_\gamma$, {\sf C}$_{{\epsilon}1}$
and {\sf N}$_{{\epsilon}2}$ are kept as normal atoms, but with masses
redistributed such that the center of mass of the ring is
preserved. See \figref{vsitehydro}D.

\end{itemize}

\section{\normindex{Viscosity} calculation}

The shear viscosity is a property of liquids that can be determined easily  
by experiment. It is useful for parameterizing a force field
because it is a kinetic property, while most other properties
which are used for parameterization are thermodynamic.
The viscosity is also an important property, since it influences
the rates of conformational changes of molecules solvated in the liquid.

The viscosity can be calculated from an equilibrium simulation using
an Einstein relation:
\beq
\eta = \frac{1}{2}\frac{V}{k_B T} \lim_{t \rightarrow \infty}
\frac{\mbox{d}}{\mbox{d} t} \left\langle 
\left( \int_{t_0}^{{t_0}+t} P_{xz}(t') \mbox{d} t' \right)^2
\right\rangle_{t_0}
\eeq
This can be done with {\tt g_energy}.
This method converges very slowly~\cite{Hess2002a}, and as such
a nanosecond simulation might not
be long enough for an accurate determination of the viscosity.
The result is very dependent on the treatment of the electrostatics.
Using a (short) cut-off results in large noise on the off-diagonal
pressure elements, which can increase the calculated viscosity by an order
of magnitude.

{\gromacs} also has a non-equilibrium method for determining
the viscosity~\cite{Hess2002a}.
This makes use of the fact that energy, which is fed into system by
external forces, is dissipated through viscous friction. The generated heat
is removed by coupling to a heat bath. For a Newtonian liquid adding a 
small force will result in a velocity gradient according to the following
equation:
\beq
a_x(z) + \frac{\eta}{\rho} \frac{\partial^2 v_x(z)}{\partial z^2} = 0
\eeq
Here we have applied an acceleration $a_x(z)$ in the $x$-direction, which
is a function of the $z$-coordinate.
In {\gromacs} the acceleration profile is:
\beq
a_x(z) = A \cos\left(\frac{2\pi z}{l_z}\right)
\eeq
where $l_z$ is the height of the box. The generated velocity profile is:
\beq
v_x(z) = V \cos\left(\frac{2\pi z}{l_z}\right)
\eeq
\beq
V = A \frac{\rho}{\eta}\left(\frac{l_z}{2\pi}\right)^2
\eeq
The viscosity can be calculated from $A$ and $V$:
\beq
\label{visc}
\eta = \frac{A}{V}\rho \left(\frac{l_z}{2\pi}\right)^2
\eeq

In the simulation $V$ is defined as:
\beq
V = \frac{\displaystyle \sum_{i=1}^N m_i v_{i,x} 2 \cos\left(\frac{2\pi z}{l_z}\right)}
         {\displaystyle \sum_{i=1}^N m_i}
\eeq
The generated velocity profile is not coupled to the heat bath. Moreover,
the velocity profile is excluded from the kinetic energy.
One would like $V$ to be as large as possible to get good statistics.
However, the shear rate should not be so high that the system gets too far
from equilibrium. The maximum shear rate occurs where the cosine is zero,
the rate being:
\beq
\mbox{sh}_{\max} =  \max_z \left| \frac{\partial v_x(z)}{\partial z} \right|
= A \frac{\rho}{\eta} \frac{l_z}{2\pi}
\eeq
For a simulation with: $\eta=10^{-3}$ [kg\,m$^{-1}$\,s$^{-1}$],
$\rho=10^3$\,[kg\,m$^{-3}$] and $l_z=2\pi$\,[nm],
$\mbox{sh}_{\max}=1$\,[ps\,nm$^{-1}$] $A$.
This shear rate should be smaller than one over the longest
correlation time in the system. For most liquids, this will be the rotation
correlation time, which is around 10 ps. In this case, $A$ should
be smaller than 0.1\,[nm\,ps$^{-2}$].
When the shear rate is too high, the observed viscosity will be too low.
Because $V$ is proportional to the square of the box height,
the optimal box is elongated in the $z$-direction.
In general, a simulation length of 100 ps is enough to obtain an
accurate value for the viscosity.

The heat generated by the viscous friction is removed by coupling to a heat
bath. Because this coupling is not instantaneous the real temperature of the
liquid will be slightly lower than the observed temperature.
Berendsen derived this temperature shift~\cite{Berendsen91}, which can
be written in terms of the shear rate as:
\beq
T_s = \frac{\eta\,\tau}{2 \rho\,C_v} \mbox{sh}_{\max}^2
\eeq
where $\tau$ is the coupling time for the Berendsen thermostat and
$C_v$ is the heat capacity. Using the values of the example above,
$\tau=10^{-13}$ [s] and $C_v=2 \cdot 10^3$\,[J kg$^{-1}$\,K$^{-1}$], we
get: $T_s=25$\,[K\,ps$^{-2}$]\,sh$_{\max}^2$. When we want the shear
rate to be smaller than $1/10$\,[ps$^{-1}$], $T_s$ is smaller than
0.25\,[K], which is negligible.

{\bf Note} that the system has to build up the velocity profile when starting
from an equilibrium state. This build-up time is of the order of the
correlation time of the liquid.

Two quantities are written to the energy file, along with their averages
and fluctuations: $V$ and $1/\eta$, as obtained from (\ref{visc}).

\section{\normindex{Tabulated interaction function}s}
\subsection{Cubic splines for potentials}
\label{subsec:cubicspline}
In some of the inner loops of {\gromacs}, look-up tables are used 
for computation of potential and forces. 
The tables are interpolated using a cubic
spline algorithm. 
There are separate tables for electrostatic, dispersion, and repulsion
interactions,
but for the sake of caching performance these have been combined
into a single array. 
The cubic spline interpolation for $x_i \leq x < x_{i+1}$ looks like this:
\beq
V_s(x) = A_0 + A_1 \,\epsilon + A_2 \,\epsilon^2 + A_3 \,\epsilon^3
\label{eqn:spline}
\eeq
where the table spacing $h$ and fraction $\epsilon$ are given by:
\bea
h	&=&	x_{i+1} - x_i	\\
\epsilon&=&	(x - x_i)/h
\eea
so that $0 \le \epsilon < 1$.
From this, we can calculate the derivative in order to determine the forces:
\beq
-V_s'(x) ~=~ 
-\frac{{\rm d}V_s(x)}{{\rm d}\epsilon}\frac{{\rm d}\epsilon}{{\rm d}x} ~=~
-(A_1 + 2 A_2 \,\epsilon + 3 A_3 \,\epsilon^2)/h
\eeq
The four coefficients are determined from the four conditions
that $V_s$ and $-V_s'$ at both ends of each interval should match
the exact potential $V$ and force $-V'$.
This results in the following errors for each interval:
\bea
|V_s  - V  |_{max} &=& V'''' \frac{h^4}{384} + O(h^5) \\
|V_s' - V' |_{max} &=& V'''' \frac{h^3}{72\sqrt{3}} + O(h^4) \\
|V_s''- V''|_{max} &=& V'''' \frac{h^2}{12}  + O(h^3)
\eea
V and V' are continuous, while V'' is the first discontinuous
derivative.
The number of points per nanometer is 500 and 2000
for single- and double-precision versions of {\gromacs}, respectively.
This means that the errors in the potential and force will usually
be smaller than the single precision accuracy.

{\gromacs} stores $A_0$, $A_1$, $A_2$ and $A_3$.
The force routines get a table with these four parameters and
a scaling factor $s$ that is equal to the number of points per nm.
({\bf Note} that $h$ is $s^{-1}$).
The algorithm goes a little something like this:
\begin{enumerate}
\item	Calculate distance vector (\ve{r}$_{ij}$) and distance r$_{ij}$
\item	Multiply r$_{ij}$ by $s$ and truncate to an integer value $n_0$
	to get a table index
\item	Calculate fractional component ($\epsilon$ = $s$r$_{ij} - n_0$) 
	and $\epsilon^2$ 
\item	Do the interpolation to calculate the potential $V$ and the scalar force $f$
\item	Calculate the vector force \ve{F} by multiplying $f$ with \ve{r}$_{ij}$
\end{enumerate}

{\bf Note} that table look-up is significantly {\em
slower} than computation of the most simple Lennard-Jones and Coulomb
interaction. However, it is much faster than the shifted Coulomb
function used in conjunction with the PPPM method. Finally, it is much
easier to modify a table for the potential (and get a graphical
representation of it) than to modify the inner loops of the MD
program.

\subsection{User-specified potential functions}
You can also use your own \swapindex{potential}{function}s 
without editing the {\gromacs} code. 
The potential function should be according to the following equation
\beq
V(r_{ij}) ~=~ \frac{q_i q_j}{4 \pi\epsilon_0} f(r_{ij}) + C_6 \,g(r_{ij}) + C_{12} \,h(r_{ij})
\eeq
where $f$, $g$, and $h$ are user defined functions. {\bf Note} that if $g(r)$ represents a
normal dispersion interaction, $g(r)$ should be $<$ 0. C$_6$, C$_{12}$
and the charges are read from the topology. Also note that combination
rules are only supported for Lennard-Jones and Buckingham, and that
your tables should match the parameters in the binary topology.

When you add the following lines in your {\tt .mdp} file:

{\small
\begin{verbatim}
rlist           = 1.0
coulombtype     = User
rcoulomb        = 1.0
vdwtype         = User
rvdw            = 1.0
\end{verbatim}}

{\tt mdrun} will read a single non-bonded table file,
or multiple when {\tt energygrp-table} is set (see below).
The name of the file(s) can be set with the {\tt mdrun} option {\tt -table}.
The table file should contain seven columns of table look-up data in the
order: $x$, $f(x)$, $-f'(x)$, $g(x)$, $-g'(x)$, $h(x)$, $-h'(x)$.
The $x$ should run from 0 to $r_c+1$ (the value of {\tt table_extension} can be
changed in the {\tt .mdp} file).
You can choose the spacing you like; for the standard tables {\gromacs}
uses a spacing of 0.002 and 0.0005 nm when you run in single
and double precision, respectively.  In this
context, $r_c$ denotes the maximum of the two cut-offs {\tt rvdw} and
{\tt rcoulomb} (see above). These variables need not be the same (and
need not be 1.0 either).  Some functions used for potentials contain a
singularity at $x = 0$, but since atoms are normally not closer to each
other than 0.1 nm, the function value at $x = 0$ is not important.
Finally, it is also
possible to combine a standard Coulomb with a modified LJ potential
(or vice versa). One then specifies {\eg} {\tt coulombtype = Cut-off} or
{\tt coulombtype = PME}, combined with {\tt vdwtype = User}.  The table file must
always contain the 7 columns however, and meaningful data (i.e. not
zeroes) must be entered in all columns.  A number of pre-built table
files can be found in the {\tt GMXLIB} directory for 6-8, 6-9, 6-10, 6-11, and 6-12
Lennard-Jones potentials combined with a normal Coulomb.

If you want to have different functional forms between different
groups of atoms, this can be set through energy groups.
Different tables can be used for non-bonded interactions between
different energy groups pairs through the {\tt .mdp} option {\tt energygrp-table}
(see \secref{mdpopt}).
Atoms that should interact with a different potential should
be put into different energy groups.
Between group pairs which are not listed in {\tt energygrp-table},
the normal user tables will be used. This makes it easy to use
a different functional form between a few types of atoms.

\section{Mixed Quantum-Classical simulation techniques}

In a molecular mechanics (MM) force field, the influence of electrons
is expressed by empirical parameters that are assigned on the basis of
experimental data, or on the basis of results from high-level quantum
chemistry calculations. These are valid for the ground state of a
given covalent structure, and the MM approximation is usually
sufficiently accurate for ground-state processes in which the overall
connectivity between the atoms in the system remains
unchanged. However, for processes in which the connectivity does
change, such as chemical reactions, or processes that involve multiple
electronic states, such as photochemical conversions, electrons can no
longer be ignored, and a quantum mechanical description is required
for at least those parts of the system in which the reaction takes
place.

One approach to the simulation of chemical reactions in solution, or
in enzymes, is to use a combination of quantum mechanics (QM) and
molecular mechanics (MM). The reacting parts of the system are treated
quantum mechanically, with the remainder being modeled using the
force field. The current version of {\gromacs} provides interfaces to
several popular Quantum Chemistry packages (MOPAC~\cite{mopac},
GAMESS-UK~\cite{gamess-uk}, Gaussian~\cite{g03} and CPMD~\cite{Car85a}).

{\gromacs} interactions between the two subsystems are
either handled as described by Field {\em et al.}~\cite{Field90a} or
within the ONIOM approach by Morokuma and coworkers~\cite{Maseras96a,
Svensson96a}.

\subsection{Overview}

Two approaches for describing the interactions between the QM and MM
subsystems are supported in this version:

\begin{enumerate}
\item{\textbf{Electronic Embedding}} The electrostatic interactions
between the electrons of the QM region and the MM atoms and between
the QM nuclei and the MM atoms are included in the Hamiltonian for
the QM subsystem: \beq H^{QM/MM} =
H^{QM}_e-\sum_i^n\sum_J^M\frac{e^2Q_J}{4\pi\epsilon_0r_{iJ}}+\sum_A^N\sum_J^M\frac{e^2Z_AQ_J}{e\pi\epsilon_0R_{AJ}},
\eeq where $n$ and $N$ are the number of electrons and nuclei in the
QM region, respectively, and $M$ is the number of charged MM
atoms. The first term on the right hand side is the original
electronic Hamiltonian of an isolated QM system. The first of the
double sums is the total electrostatic interaction between the QM
electrons and the MM atoms. The total electrostatic interaction of the
QM nuclei with the MM atoms is given by the second double sum. Bonded
interactions between QM and MM atoms are described at the MM level by
the appropriate force field terms. Chemical bonds that connect the two
subsystems are capped by a hydrogen atom to complete the valence of
the QM region. The force on this atom, which is present in the QM
region only, is distributed over the two atoms of the bond. The cap
atom is usually referred to as a link atom.

\item{\textbf{ONIOM}} In the ONIOM approach, the energy and gradients
are first evaluated for the isolated QM subsystem at the desired level
of {\it{ab initio}} theory. Subsequently, the energy and gradients of
the total system, including the QM region, are computed using the
molecular mechanics force field and added to the energy and gradients
calculated for the isolated QM subsystem. Finally, in order to correct
for counting the interactions inside the QM region twice, a molecular
mechanics calculation is performed on the isolated QM subsystem and
the energy and gradients are subtracted. This leads to the following
expression for the total QM/MM energy (and gradients likewise): \beq
E_{tot} = E_{I}^{QM}
+E_{I+II}^{MM}-E_{I}^{MM}, \eeq where the
subscripts I and II refer to the QM and MM subsystems,
respectively. The superscripts indicate at what level of theory the
energies are computed. The ONIOM scheme has the
advantage that it is not restricted to a two-layer QM/MM description,
but can easily handle more than two layers, with each layer described
at a different level of theory.
\end{enumerate}

\subsection{Usage}

To make use of the QM/MM functionality in {\gromacs}, one needs to:

\begin{enumerate}
\item introduce link atoms at the QM/MM boundary, if needed;
\item specify which atoms are to be treated at a QM level;
\item specify the QM level, basis set, type of QM/MM interface and so on. 
\end{enumerate}

\subsubsection{Adding link atoms}

At the bond that connects the QM and MM subsystems, a link atoms is
introduced.  In {\gromacs} the link atom has special atomtype, called
LA. This atomtype is treated as a hydrogen atom in the QM calculation,
and as a virtual site in the force field calculation. The link atoms, if
any, are part of the system, but have no interaction with any other
atom, except that the QM force working on it is distributed over the
two atoms of the bond. In the topology, the link atom (LA), therefore,
is defined as a virtual site atom:

{\small
\begin{verbatim}
[ virtual_sites2 ]
LA QMatom MMatom 1 0.65
\end{verbatim}}

See~\secref{vsitetop} for more details on how virtual sites are
treated. The link atom is replaced at every step of the simulation.

In addition, the bond itself is replaced by a constraint:

{\small
\begin{verbatim}
[ constraints ]
QMatom MMatom 2 0.153
\end{verbatim}}

{\bf Note} that, because in our system the QM/MM bond is a carbon-carbon
bond (0.153 nm), we use a constraint length of 0.153 nm, and dummy
position of 0.65. The latter is the ratio between the ideal C-H
bond length and the ideal C-C bond length. With this ratio, the link
atom is always 0.1 nm away from the {\tt QMatom}, consistent with the
carbon-hydrogen bond length. If the QM and MM subsystems are connected
by a different kind of bond, a different constraint and a different
dummy position, appropriate for that bond type, are required.

\subsubsection{Specifying the QM atoms}

Atoms that should be treated at a QM level of theory, including the
link atoms, are added to the index file. In addition, the chemical
bonds between the atoms in the QM region are to be defined as
connect bonds (bond type 5) in the topology file:

{\small
\begin{verbatim}
[ bonds ]
QMatom1 QMatom2 5
QMatom2 QMatom3 5
\end{verbatim}}

\subsubsection{Specifying the QM/MM simulation parameters}

In the {\tt .mdp} file, the following parameters control a QM/MM simulation.

\begin{description}

\item[\tt QMMM = no]\mbox{}\\ If this is set to {\tt yes}, a QM/MM
simulation is requested. Several groups of atoms can be described at
different QM levels separately. These are specified in the QMMM-grps
field separated by spaces. The level of {\it{ab initio}} theory at which the
groups are described is specified by {\tt QMmethod} and {\tt QMbasis}
Fields. Describing the groups at different levels of theory is only
possible with the ONIOM QM/MM scheme, specified by {\tt QMMMscheme}.

\item[\tt QMMM-grps =]\mbox{}\\groups to be described at the QM level

\item[\tt QMMMscheme = normal]\mbox{}\\Options are {\tt normal} and
{\tt ONIOM}. This selects the QM/MM interface. {\tt normal} implies
that the QM subsystem is electronically embedded in the MM
subsystem. There can only be one {\tt QMMM-grps} that is modeled at
the {\tt QMmethod} and {\tt QMbasis} level of {\it{ ab initio}}
theory. The rest of the system is described at the MM level. The QM
and MM subsystems interact as follows: MM point charges are included
in the QM one-electron Hamiltonian and all Lennard-Jones interactions
are described at the MM level. If {\tt ONIOM} is selected, the
interaction between the subsystem is described using the ONIOM method
by Morokuma and co-workers. There can be more than one QMMM-grps each
modeled at a different level of QM theory (QMmethod and QMbasis).

\item[\tt QMmethod = ]\mbox{}\\Method used to compute the energy
and gradients on the QM atoms. Available methods are AM1, PM3, RHF,
UHF, DFT, B3LYP, MP2, CASSCF, MMVB and CPMD. For CASSCF, the number of
electrons and orbitals included in the active space is specified by
{\tt CASelectrons} and {\tt CASorbitals}. For CPMD, the plane-wave
cut-off is specified by the {\tt planewavecutoff} keyword.

\item[\tt QMbasis = ]\mbox{}\\Gaussian basis set used to expand the
electronic wave-function. Only Gaussian basis sets are currently
available, i.e. STO-3G, 3-21G, 3-21G*, 3-21+G*, 6-21G, 6-31G, 6-31G*,
6-31+G*, and 6-311G. For CPMD, which uses plane wave expansion rather
than atom-centered basis functions, the {\tt planewavecutoff} keyword
controls the plane wave expansion.

\item[\tt QMcharge = ]\mbox{}\\The total charge in {\it{e}} of the {\tt
QMMM-grps}. In case there are more than one {\tt QMMM-grps}, the total
charge of each ONIOM layer needs to be specified separately.

\item[\tt QMmult = ]\mbox{}\\The multiplicity of the {\tt
QMMM-grps}. In case there are more than one {\tt QMMM-grps}, the
multiplicity of each ONIOM layer needs to be specified separately.

\item[\tt CASorbitals = ]\mbox{}\\The number of orbitals to be
included in the active space when doing a CASSCF computation.

\item[\tt CASelectrons = ]\mbox{}\\The number of electrons to be
included in the active space when doing a CASSCF computation.

\item[\tt SH = no]\mbox{}\\If this is set to yes, a QM/MM MD
simulation on the excited state-potential energy surface and enforce a
diabatic hop to the ground-state when the system hits the conical
intersection hyperline in the course the simulation. This option only
works in combination with the CASSCF method.

\end{description}

\subsection{Output}

The energies and gradients computed in the QM calculation are added to
those computed by {\gromacs}. In the {\tt .edr} file there is a section
for the total QM energy.

\subsection{Future developments}

Several features are currently under development to increase the
accuracy of the QM/MM interface. One useful feature is the use of
delocalized MM charges in the QM computations. The most important
benefit of using such smeared-out charges is that the Coulombic
potential has a finite value at interatomic distances. In the point
charge representation, the partially-charged MM atoms close to the QM
region tend to ``over-polarize'' the QM system, which leads to artifacts
in the calculation.

What is needed as well is a transition state optimizer.

\section{{\gromacs} on GPUs}

{\gromacs} 4.5 provides support for GPU acceleration through 
the \href{https://simtk.org/home/openmm}{OpenMM library}. 
Although limited in functionality compared to the CPU algorithms of standard {\gromacs},
{\gromacs}-GPU gives a preview of GPU-accelerated MD which we expect to be an important direction in the future. 
The following should be noted before using the accelerated {\tt mdrun-gpu}:
\begin{itemize}
\item The current release runs only on modern, CUDA compatible NVIDIA GPU hardware 
(for details see section~\ref{subsec:compatibility}). Make sure that the necessary CUDA drivers and 
libraries for your operating system are already installed. 
\item Only single-GPU simulations are supported.
\item Only a relatively small subset of the {\gromacs} features and options are supported 
with the  current OpenMM-based implementation. See section~\ref{subsec:features} for 
a detailed list.
\item Consumer-level GPU cards are known to often have problems with faulty memory.
It is strongly recommended that a full memory check of the cards is done at least once
(using option {\tt memtest=full}). A partial memory check (using option {\tt memtest=15}) 
before and after the simulation run would help spot problems resulting from overheating 
of the graphics card.
\item The maximum size of the simulated systems depends on the GPU used. With
high-end cards, like GeForce GTX 480 or Tesla 2050, systems of size up to 200,000 atoms 
have been successfully simulated.
\item In order to take a full advantage of the GPU platform features, many algorithms
have been implemented in a very different way than they are on the CPUs.
Therefore, numerical correspondence between some properties of the system's state
should not be expected. Moreover, the values will likely vary when simulations are
done on different hardware. However, sufficiently long trajectories
should produce comparable statistical averages.
\item Frequent retrieval of system state information such as
trajectory coordinates and energies can greatly influence the performance
of the program due the overhead of CPU$\leftrightarrow$GPU communication.
\item MD algorithms are complex and often do not translate very well onto streaming architectures.
Realistic expectations about the achievable speed-up from tests with a GeForce 400 series card:
for small protein systems in implicit solvent using all-vs-all kernels the speedup 
can be up to 10-15x, but in most other setups involving cutoffs and PME, 
performance is close to the one on a modern 4 core CPU (e.g. Intel Core i7-930).
\end{itemize}

\subsection{Supported features}\label{subsec:features}

\begin{itemize}
\item \textbf{Integrators:} {\tt md/md-vv/md-vv-avek}, {\tt sd/sd1} and {\tt bd}.\\
OpenMM implements only the velocity-Verlet algorithm for MD simulations.
The option {\tt md} is accepted, but keep in mind that the actual algorithm is \emph{not} leap-frog.
Thus all three options {\tt md}, {\tt md-vv}, and {\tt md-vv-avek} are equivalent.
Similarly, {\tt sd} and {\tt sd1} are equivalent.
\item \textbf{Long-range interactions:} {\tt Reaction-Field}, {\tt Ewald}, {\tt PME}, {\tt No-cutoff}
{\tt Cut-off}.
\begin{itemize}
\item for No-cutoff, set {\tt rcoulomb=0} and {\tt rvdw=0}.
\item for Ewald summation only 3D geometry is supported, and dipole correction is not.
\item the Cut-off method is supported only for implicit solvent simulations.
\end{itemize}
\item \textbf{Temperature control:} Supported only with the {\tt sd/sd1}, {\tt bd},
{\tt md/md-vv/md-vv-avek} integrators.
OpenMM implements only the Andersen thermostat. All values for {\tt tcoupl} are
thus accepted and equivalent to {\tt andersen}. Multiple temperature coupling groups
are not supported, only {\tt tc-grps=System} will work.
\item \textbf{Force fields:} Supported FF are Amber, CHARMM. {\gromos} and OPLS-AA are not supported.
\begin{itemize}
\item CMAP dihedrals in CHARMM are not supported, so use the {\tt -nocmap} option with {\tt pdb2gmx}.
\end{itemize}
\item \textbf{Implicit solvent:} Supported only with {\tt Reaction-Field} electrostatics.
The only supported algorithm for GB is OBC, and the default {\gromacs} values for the scale fators
are hard coded in OpenMM, {\ie} {\tt obc_alpha=1}, {\tt obc_beta=0.8} and {\tt obc_gamma=4.85} and 
therefore can not be changed.
\item \textbf{Constraints:} Constraints in OpenMM are done by a combination of SHAKE,
SETTLE and CCMA. Accuracy is based on the SHAKE tolerance as set by the {\tt shake_tol} option.
\item \textbf{Periodic Boundary Conditions:} Only {\tt pbc=xyz} and {\tt pbc=no} in rectangular cells (boxes) are supported.
\item \textbf{Pressure control:} OpenMM implements the Monte Carlo barostat. All values for {\tt pcoupl} are thus accepted.
\item \textbf{Simulated annealing:} Not supported.
\item \textbf{Pulling:} Not supported. 
\item \textbf{Restraints:} Distance, orientation, angle and dihedral restraints are not supported
in the current implementation.
\item \textbf{Free energy calculations:} Not supported in the current implementation.
\item \textbf{Walls:} Not supported.
\item \textbf{Non-equilibrium MD:} Option {\tt acc_grps} is not supported.
\item \textbf{Electric Fields:} Not supported.
\item \textbf{QMMM:} Not supported.
\end{itemize}

\subsection{Installing and running {\gromacs}-GPU}

{\gromacs}-GPU can be installed either from the officially distributed binary or source packages. 
We provide pre-compiled binaries built for and tested on the most common Linux, Windows, 
and Mac OS operating systems (for details see the {\gromacs}-GPU 
\href{http://www.gromacs.org/gpu}{download page}). 
Using the binary distribution is highly recommended and it should 
work in most of the cases. Below we summarize how to get the GPU accelerated {\tt mdrun-gpu} work.

\subsubsection{Prerequisites}

The current {\gromacs}-GPU release uses \href{https://simtk.org/home/openmm}{OpenMM} 
acceleration, the necessary libraries and plug-ins are included in the binary 
packages. 

Both the OpenMM library and {\gromacs}-GPU require version 3.1 of
the CUDA libraries and compatible NVIDIA driver (i.e. version $>$ 256). 

Last but not least, to run GPU accelerated simulations, a CUDA-enabled 
graphics card is necessary. Molecular dynamics algorithms 
are very demanding and unlike in other application areas, 
only high-end graphics cards are capable of providing performance comparable 
to or higher then modern CPUs. For this reason, {\tt mdrun-gpu} is compatible with 
only a subset of more performant CUDA-enabled GPUs (for detailed list see section 
\ref{subsec:compatibility}) and by default it does not run if it detects
incompatible hardware. 

For details about compatibility of NVIDIA drivers with harthe CUDA library and GPUs consult  
the \href{http://developer.nvidia.com/object/gpucomputing.html}{NVIDIA developer page}.

Summary of prerequisites: 
\begin{itemize}
    \item OpenMM v2.0
    \item NVIDIA CUDA libraries v3.1
    \item NVIDIA drivers $\geq$v256
    \item NVIDIA CUDA-enabled GPU
\end{itemize}


\subsubsection{Installing}

\begin{enumerate}

    \item Download and unpack the binary package for the respective OS and architecture. 
    Copy the content of the package to your normal {\gromacs} installation directory (or to a custom location).
    
    {\bf Note} that the distributed {\gromacs}-GPU packages do not contain the entire set of 
    tools and utilities included in a full {\gromacs} installation. Therefore, it is recommended to  
    have a $\geq$v4.5 standard {\gromacs} installation along the GPU accelerated one.

    \item Add the {\tt openmm/lib} directory to your library path,  {\eg} in bash:\\
    {\small{\tt export LD_LIBRARARY_PATH=path_to_gromacs/openmm/lib:\$LD_LIBRARY_PATH}}\\
    If there are other OpenMM versions installed, make sure that the supplied libraries have preference  
    when running {\tt mdrun-gpu}. Also, make sure that the CUDA libraries installed match the version 
    of CUDA that was used for compilation of {\gromacs}-GPU.

    \item Set the {\tt OPENMM_PLUGIN_DIR} environment variable to contain the path to the \\ 
    {\tt openmm/lib/plugins} directory, {\eg} in bash:\\ 
    {\small{\tt export OPENMM_PLUGIN_DIR=path_to_gromacs/openmm/lib/plugins}}

    \item At this point, running the command
    {\tt path_to_gromacs/bin/mdrun-gpu -h} should display the standard {\tt mdrun-gpu} help  
    which means that the binary runs and all the necessary libraries are accessible. 

\end{enumerate}

\subsubsection{Compiling {\tt mdrun-gpu}}

The GPU accelerated {\tt mdrun} can be compiled on Linux, Mac OS and Windows 
operating systems, both for 32- and 64-bit. Besides the prerequisites 
discussed above, in order to compile {\tt mdrun-gpu} the following additional 
software is required: 
\begin{itemize}
    \item CMake version $\geq$2.6.4
    \item CUDA-compatible compiler: 
        \begin{itemize}
            \item MSVC 8 or 9 on Windows 
            \item gcc 4.4 on Linux and Mac OS
        \end{itemize}
    \item CUDA toolkit 3.1
    \item OpenMM-2.0 header files
\end{itemize}
{\bf Note} that the current {\gromacs}-GPU release is compatible with OpenMM  
version 2.0. While future versions might be compatible, using the officially-supported 
and tested OpenMM version is strongly encouraged.
OpenMM binaries as well as source code can be obtained from the 
\href{https://simtk.org/project/xml/downloads.xml?group_id=161}{project's homepage}.

Also note that it is essential that the same version of CUDA is used to compile both
{\tt mdrun-gpu} and the OpenMM libraries. 

To compile {\tt mdrun-gpu} change to the top level directory of the source 
tree and execute the following commands: 

{\small
\begin{verbatim}
    export OPENMM_ROOT_DIR=path_to_custom_openmm_installpath
    cmake  -DGMX_OPENMM=ON [-DCMAKE_INSTALL_PREFIX=desired_install_path]
    make mdrun
    make install-mdrun
\end{verbatim}}


%\subsubsection{Testing and troubleshooting}

\subsubsection{{\gromacs}-GPU specific {\tt mdrun} features}

Besides the usual command line options, 
{\tt mdrun-gpu} also supports a set of ``device options'', that are meant to 
give control over acceleration related functionalities. 
These options can be used in the following form:

{\small {\tt mdrun-gpu -device "ACCELERATION:[DEV_OPTION=VALUE,]... [OPTION].."}}

The option-list prefix {\tt ACCELERATION} specifies which 
acceleration library should be used. At the moment, the only supported value 
is {\tt OpenMM}. This is followed by a list of comma-separated
{\tt DEV_OPTION=VALUE} option-value pairs which define 
parameters for the selected acceleration platform. The entire device option 
string is \emph{case insensitive}.

Below we summarize the available options (of the OpenMM acceleration library)
and their possible values.

\paragraph{Platform} Selects the GPGPU platform to be used. Currently the only 
supported value is CUDA.  In the future, OpenCL support will be added.

\paragraph{DeviceID} The numeric identifier of the CUDA device on which the 
simulation will be carried out. The default value is 0, {\ie} the first device. 

\paragraph{Memtest} GPUs, especially consumer-level devices, 
are prone to memory errors. There might be various reasons that 
``soft errors'' happen, including (factory) overclocking, overheating, 
faulty hardware etc, but the result is always the same: unreliable, possibly 
incorrect results. Therefore, {\tt mdrun-gpu} has a built-in mechanism for 
testing the GPU memory in order to catch the obviously faulty hardware. 
A set of tests are performed before and after each simulation, and if
errors are detected, the execution is aborted. 

Accepted values for this option are any integer $\leq$15 with an optional ``s'' 
suffix representing the amount of time in seconds that should be spent 
on testing; the default value is {\tt memtest=15s} (the actual time spent 
on testing might differ slightly).
It is possible to completely turn off memory testing by setting 
{\tt memtest=off}, however this is not advisable.

\paragraph{Force-device} Enables running {\tt mdrun-gpu} on
devices that are not supported but CUDA-capable. Using 
this option might results in very low performance or even 
crashes and therefore it is not encouraged.

\subsection{Hardware and software compatibility list}\label{subsec:compatibility}

Compatible OpenMM versions:
\begin{itemize}
\item v2.0
\end{itemize}

Compatible NVIDIA CUDA versions (also see OpenMM version compatibility above): 
\begin{itemize}
\item v3.1
\end{itemize}

Compatible hardware (for further details consult the 
\href{http://www.nvidia.com/object/cuda\_gpus.html}{NVIDIA CUDA compatible GPUs page}):
\begin{itemize}
\item G92/G94:
    \begin{itemize}
    \item GeForce 9800 GX2/GTX/GTX+/GT
    \item GeForce 9800M GT
    \item GeForce GTS 150, 250
    \item GeForce GTX 280M, 285M
    \item Quadro FX 4700
    \item Quadro Plex 2100 D4
    \end{itemize}
\item GT200:
    \begin{itemize}
    \item GeForce GTX 260, 270, 280, 285, 295 
    \item Tesla C1060, S1070, M1060
    \item Quadro FX 4800, 5800
    \item Quadro CX
    \item Quadro Plex 2200 D2, 2200 S4
    \end{itemize}
\item GF10x (Fermi):
    \begin{itemize}
    \item GeForce GTX 460, 465, 470, 480       
    \item Tesla C2050, C2070, S2050, M2050, M2070
    \item Quadro 5000(M), 6000
    \end{itemize}
\end{itemize}

\section{Using VMD plug-ins for trajectory file I/O}
\index{VMD plug-ins}{\gromacs} tools are able to use the plug-ins found
in an existing installation of
\href{http://www.ks.uiuc.edu/Research/vmd}{VMD} in order to read and
write trajectory files in formats that are not native to
{\gromacs}. You will be able to supply an AMBER DCD-format trajectory
filename directly to GROMACS tools, for example.

This requires a VMD installation not older than version 1.8, that your
system provides the dlopen function so that programs can determine at
run time what plug-ins exist, and that you build shared libraries when
building GROMACS. CMake will find the vmd executable in your path, and
from it, or the environment variable {\tt VMDDIR} at configuration or
run time, locate the plug-ins. Alternatively, the {\tt VMD_PLUGIN_PATH}
can be used at run time to specify a path where these plug-ins can be
found. Note that these plug-ins are in a binary format, and that format
must match the architecture of the machine attempting to use them.

% LocalWords:  PMF pmf kJ mol Jarzynski BT bilayer rup mdp AFM fepmf fecalc rb
% LocalWords:  posre grompp fs Verlet dihedrals hydrogens hydroxyl sulfhydryl
% LocalWords:  vsitehydro planarity chirality pdb gmx virtualize virtualized xz
% LocalWords:  vis massless tryptophan histidine phenyl parameterizing ij PPPM
% LocalWords:  parameterization Berendsen rlist coulombtype rcoulomb vdwtype LJ
% LocalWords:  rvdw energygrp mdrun pre GMXLIB mdpopt MOPAC GAMESS CPMD ONIOM
% LocalWords:  Morokuma iJ AQ AJ initio atomtype QMatom MMatom QMMM grps et al
% LocalWords:  QMmethod QMbasis QMMMscheme RHF DFT LYP CASSCF MMVB CASelectrons
% LocalWords:  CASorbitals planewavecutoff STO QMcharge QMmult diabatic edr GPU
% LocalWords:  hyperline delocalized Coulombic GPUs OpenMM NVIDIA CUDA memtest
% LocalWords:  GTX CPUs GHz md sd bd vv avek tcoupl andersen tc OPLSAA GROMOS
% LocalWords:  OBC obc CCMA tol pbc xyz barostat pcoupl acc gpu PLUGIN Cmake GX
% LocalWords:  MSVC gcc installpath cmake DGMX DCMAKE functionalities GPGPU GTS
% LocalWords:  OpenCL DeviceID gromacs gpus html GeForce Quadro FX Plex CX GF
% LocalWords:  VMD DCD
